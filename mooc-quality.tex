\section{Quality of a MOOC}

%% section editor: David Glance

\tbd{edX has been working with QualityMatters to develop a strawman
 rubric for assessing MOOC quality.  We shoudl take a look and/or cite
 it.}


Although the question of quality of a MOOC should, in principle, be no
different than the question of quality applied to any course produced by
an educational provider, MOOCs have additional properties that make the
question of MOOC quality unique. Firstly, MOOCs are usually free or very
little cost to the student. Second, the massive scale has meant that the
offering institutions have seen MOOCs as a vehicle for publicity and
reputational enhancement as much as an educational device.

\subsection{Several Dimensions of Quality}

There are several dimensions to Quality that depend largely on the
perspective of the actor involved. The ultimate arbiter of this
perception however is the student taking the MOOC and this makes it
possibly more difficult to pin down as the students are not a homogenous
group and their expectations will be different. This is especially true
if fees are associated with MOOCs.  Having said that, the dimensions of
MOOC quality will include:

\begin{enumerate}

\item Learning effectiveness, fulfilment of learning objectives 

\item Engaging students

\item Carried out professionally

\item Having a particular quality of student cohort

\end{enumerate}


Note that (1) and (2) may be in conflict and may receive different
emphasis from different groups. For instance, university administrators
may want to ensure that there is higher retention and reputational
outcome which emphasises engagement and few challenges whilst the MOOC
creator may want to emphasise pedagogical integrity which pushes the
students, even beyond their comfort zone.

A particular quality risk with MOOCs is to provide overpolished,
entertaining video-material with expensive, professionally animated
graphics, i.e. focussing on (2) and (3), while paying less attention to
the actual learning objectives, i.e. (1)---which might or might not have
been defined in the first place anyhow. In general, designers of MOOC
courses and platforms should first think about the pedagogy of their
course and after that worry about the technology that is required to
implement their pedagogy. Pedagogy should drive technology and not vice
versa.

As mentioned before, the specific properties that determine quality of
MOOCs are context dependent. The MOOC style (cMOOC vs xMOOC vs any form
of blended learning) will determine different aspects of quality placing
perhaps a greater emphasis on student cohort quality, e.g. students in a
cMOOC might be more engaged than students in an xMOOC.


\subsection{Existing Quality Problems}

The discussion of quality can already be observed in some current
examples MOOC offerings examples of which could be considered poor
quality. For instance:

\begin{enumerate}

\item badly designed videos, e.g. poor presentation, poor or inappropriate
material, poor quizzes

\item MOOC not based on any pedagogical
underpinning 

\item MOOC not clear or accurate about learning goals and
outcomes

\item technical problems: poor sound, unreadable slides, poor
video quality or delivery, audio and video out of sync

\end{enumerate}




Most universities already have processes ensuring their traditional
course offerings meet particular quality criteria (accreditation). Thus,
these processes should be extended to include MOOC offerings as
well. However, in addition, there is a question regarding whether
official certifying bodies such as ACM for example, may want to certify
course offerings as well.
