%\section{Quality of a MOOC}
\section{Quality criteria for MOOCs should be developed}

%% section editor: David Glance

Although the question of quality of a MOOC should, in principle, be no
different than the question of quality applied to any course produced by
an educational provider, MOOCs have additional properties that make the
question of MOOC quality unique. First, MOOCs are usually free or very
low cost to the learner. Second, the massive scale has meant that the
offering institutions have seen MOOCs as a vehicle for publicity and
reputational enhancement as much as an educational device.

\tbd{comment by fb: This section opening is fine - but it reflects ony the US and UK perspectives and it misses a
  central element.

  A continental European persoective should address the issue of the filtering of those students
  well capable of studying in the fields they have chosen.

  MOOCs hold the promise to detect those students from the emerging countries who (1) yet would not
  get grants and residence permits for studying in a de velopped country but (2) have the skills and
  dedication to do so. MOOCs will act a gates to face-to-face higher education for such students. }


\subsection{Dimensions of Quality Depend on Context and Actor}

Several dimensions of quality depend largely on the
MOOC's context, the perspective of a particular actor, or both.
For example, one important element is the perception of the MOOC by
enrolled learners,
%% The ultimate arbiter of this
%% perception however is the student taking the MOOC and this makes it
%% possibly more 
which is difficult to pin down as the learners are a heterogeneous group
with varying expectations. 
The learners' perceptions are even more important 
if fees are associated with MOOCs.  
As another example, the MOOC style (cMOOC vs.\ xMOOC vs.\
blended learning) may determine whether greater emphasis is placed on
learner cohort quality; for example, learners in a
cMOOC might be more engaged than learners in an xMOOC.

Other dimensions of
MOOC quality include the following:

\begin{enumerate}

\item Learning effectiveness, fulfilment of learning objectives;

\item Engagement of learners;

\item Professionalism of preparation and execution;

\item Quality of the learner cohort.

\end{enumerate}

\tbd{comment by jh: i was puzzled by this and found it hard to follow.  end of the para says '...might be more
  engaged...' then the bullets says other dimensions include engagement.  Bullet 3. is clear,
  4. could be if we define quality of the cohort, maybe 1. means 'extent of student attainment of
  learning outcomes'??

  I find it a bit difficult to follow the logic of 1 and 2 being in conflict.  Wont more engaged
  learners generally achieve better results??  
 ck: The idea here is to present different quality criteria that may indeed not be all consistent:
 so I don't see the point of Jeff and propose to leave the text as it is.
}

Note that (1) and (2) may be in conflict and may receive different
emphasis from different groups. For instance, university administrators
may want to emphasise retention and reputational
outcome, leading to an emphasis on engagement and few challenges,
while the MOOC
creator may want to emphasise pedagogical integrity, which may push the
learners beyond their comfort zone.

A particular quality risk with MOOCs is to focus on points (2) and (3)
at the expense of point (1): that is, providing overpolished,
entertaining video-material with expensive, professionally-animated
graphics, while paying less attention to
the actual learning objectives, which might or might not have 
been properly defined in the first place. In general, designers of MOOC
courses and platforms should first think about the pedagogy of their
course and after that worry about the technology that is required to
implement their pedagogy. Pedagogy should drive technology and not vice
versa.


\subsection{Existing Quality Problems}

Some current MOOC offerings have already grappled with problems that
could be considered issues of quality, for example:
%% The discussion of quality can already be observed in some current
%% MOOC offerings, some examples of which could be considered poor
%% quality. For instance:

\begin{enumerate}

\item Video/content quality problems: poor presentation, poor or inappropriate
material, poor quizzes

\item MOOC not based on any pedagogical
underpinning 

\item MOOC not clear or accurate about learning goals and
outcomes

\item Technical problems: poor sound, unreadable slides, poor
video quality or delivery, audio and video out of sync, learner-facing
technology that collapses at large scale or is a poor fit to the teaching material.

\end{enumerate}

Most universities already have processes ensuring their traditional
course offerings meet particular quality criteria, both for
accreditation and to maintain internal standards.
These processes should be extended to include MOOC offerings as well.
Even if they do so, however, official certifying bodies, such as the
Association for Computing Machinery, may want to certify course
offerings as well.
For example, the Quality Matters (QM) Program
(\url{http://qualitymatters.org}) describes itself as ``an
international organization representing broad inter-institutional
collaboration and a shared understanding of online course quality.''
A recent Gates Foundation program to fund MOOC development required
the creators to self-evaluate their MOOCs using the QM rubric, which
measures specific course properties (such as whether learning goals
are clearly stated) rather than relying on learners' subjective
appraisals of the MOOC.
A summary of the initial results of this
review~\cite{gates-qm-review-summary-2013} showed that while most
supported MOOCs did fairly well in meeting the quality criteria, the
most common areas of weakness were institutional responsibilities that
are quite different in a MOOC than in a residential course, such as
articulation of course support services and of policies regarding
accessibility issues.  As well, a challenge reported by many of the
reviewers was the mixture of for-credit and non-for-credit learners in
MOOCs---two cohorts with different sets of expectations.
In Europe, metrics for assessing quality of online learning provision have been in existence for
several years, and EFQUEL has recently extended theirs to encompass MOOCs (\url{http://efquel.org}).
