\section{Research questions}
Today a MOOC  is primarily organized around lectures by a university instructor; it is taught as an
xMOOC in a traditional lecture-oriented format. One consequence is that MOOCs have a rather rigid
structure and suffer from high drop-out rates.  Future MOOCs should provide better learning
experiences; they should be more enjoyable, adaptive, evaluable, accessible, usable by handicapped
learners, aware of cultural diversity and ethical context, and so on.

These requirements create new research problems and new methods to address some of them. We briefly present some example research questions around these issues; this list is not meant to be exhaustive.  

\medskip
MOOC platforms should provide computer support for a large variety of pedagogical methods and should be accessible to the broad diversity of learners. This leads to two initial sets of research questions:
 
\begin{itemize}
\item \textbf{Better support for pedagogical methods:}
\begin{description}
\item[Responsive:] more research is needed to give the learners more
  feedback, such as via machine grading or peer grading of exercises,
  and also to evaluate their success and deliver certificates. 
\item[Adapting to the learner:] in classroom pedagogy, there is little room for offering each learner a particular sequence of exercises adapted to his level and difficulties. With MOOCs, such adaptation becomes possible, but research is needed to understand which algorithms foster this adaptation.
\item[Enjoyable:] children learn much by playing games; learning itself
  is also often an enjoyable activity. These observations suggest
  investigating how courses could be ``gamified'', or transformed into educational games.
\item[Pedagogy at scale:] more generally, research is needed for new pedagogical approaches that are intrinsically designed to take advantage of scale.
Pedagogical activities initially designed for small scale should be
examined for the possibility of 
adapting to massive numbers of learners, and  opportunities for novel
scale-powered pedagogical activities, such as direct learner-to-learner communication, should be investigated.
\end{description}
\item \textbf{Better support for learner diversity:}
\begin{description}
\item[Accessible:] current MOOC platforms are difficult to use by deaf, blind, and other disabled learners. More research is needed to understand how to make them accessible.
\item[Usable by underserved learners:] MOOCs have been presented as a
  possible way to improve literacy rates and school retention for
  underserved learners, but such MOOCs must use strategies, interfaces,
  and learning techniques appropriate for these specific learners.
\item[Aware of cultural diversity:] offering courses in different languages is essential for offering education to everyone. Besides the diversity of languages, more research is needed to present the courses and the exercises in a way that is compatible with the different (academic) cultures.


  % \item[??diversity:] not all courses are organized around the talk of a professor. More research is needed to understand how to make MOOCs more diverse, by integrating multi-modal user interfaces, remote labs, etc. in order to implement active pedagogy in MOOCs.
  % \item[Enhancing pedagogy] help teachers to improve quality of MOOCs.
\end{description}
\end{itemize}

The history of any science can be divided into two eras separated by the
appearance of effective measuring instruments.
Astronomy began with the naked eye but has had good instruments since
Galileo's telescope.
Biology has been instrumented since van~Leeuwenhoek's microscope.
Mathematics has been instrumented only since the middle of the 1970s,
when a computer was used to help prove the four-color theorem. If the
computer is the mathematician's telescope, the network may be the social
scientist's telescope, as it affords the observation of social
interaction on a large scale. This is particularly the case with
pedagogy: MOOCs afford observation at large scale of how people teach,
how they learn, and so on. Thus MOOC platforms can serve as a novel
measuring instrument in pedagogy. On the other hand, access to detailed
MOOCs data traces requires respecting the personal rights of learners
and teachers.
Research is needed on the ethics as well as the algorithms of
evaluating the quality of a MOOC and measuring the activities of its
learners:

\begin{itemize}
\item \textbf{Better analytics of course data:}
\begin{description}
\item[Evaluating quality:] more research is needed to evaluate the
  quality of a MOOC: Besides the number of learners, the drop-out rate, and the reputation of the course, factors such as learning effectiveness, fulfilment of learning objectives, or the  quality of the student cohort must be investigated.
\item[Measuring learner activities:] developing instrumented pedagogy
  requires improving  learning analytics; for example, analyzing the data traces of the learners in order to understand and predict the  progress of learners or to help teachers in planning supporting interventions and in improving the content of a course.
\item[Ethical:] more research on the ethical use of MOOCs data is needed
  to understand which kind of information should be accessible to whom,
  focusing on the privacy issues related to the data produced by the
  learners. Besides defining goals, research is also needed to
  understand their implementation in MOOC platforms, requiring, for
  instance, the development of privacy models and methods to develop
  MOOC platforms so that privacy is enforced by design.
\end{description}
\end{itemize}

Finally, more research is needed to understand the societal and economic
impact of MOOCs. Some interesting research questions include how
to transform on-line communities into geographically confined
ones, and what MOOC business models will balance the efforts of private industry and
public services. 


%%% Local Variables: 
%%% mode: latex
%%% TeX-master: "manifesto"
%%% End: 
