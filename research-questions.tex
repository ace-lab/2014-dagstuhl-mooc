\section{Research questions}
%The development of MOOCs creates new research problems and new methods to address some of them. The new research problems come from the will to make MOOC platforms and MOOCs more diverse, ethical, enjoyable, adaptive, evaluable, accessible, usable by underserved learners, aware of cultural diversity, etc. 

Today a MOOC  is mostly organized around a talk of a university teacher; it is taught as an xMOOC in a traditional lecture-oriented format. One of the consequences is that MOOCs have a rather rigid structure and suffer from high drop-out rates.  Future MOOCs should provide better learning experiences; they should be more enjoyable, adaptive, evaluable, accessible, usable by handicapped learners, aware of cultural diversity, ethical etc.

These requirements create new research problems and new methods to address some of them. In the following we shortly present some exemplary research questions around these issues; this list is not meant to be exhaustive.  

\medskip
MOOC platforms should provide computer support for a large variety of pedagogical methods and should be accessible to the broad diversity of learners. This leads to two first sets of research questions:

\begin{itemize}
\item \textbf{Better support for pedagogical methods:}
\begin{description}
\item[Responsive:] more research is needed to give the learners more feedback, by correcting exercises automatically or by the peers, and also to evaluate their success and deliver diplomas. 
\item[Adapting to learner:] in classroom pedagogy, there is little room for offering each learner a particular path through knowledge: a particular sequence of exercises adapted to the level and the difficulties of each student. With MOOCs, such an adaptation to the learner becomes possible, but research is needed to understand which algorithms foster this adaptation.
\item[Enjoyable:] children learn a lot by playing games; learning itself is also often an enjoyable activity. These two remarks lead to the idea to investigate how courses could be gamified, that is transformed into educational games.
\item[Pedagogy at scale:] more generally, research is needed for new pedagogical approaches that are intrinsically designed to take advantage of scale.
Pedagogical activities initially designed for low scale have to be adapted to massive numbers of learners and  opportunities for novel scale-powered pedagogical activities, e.g. in direct learner-to-learner communication, should be investigated.
\end{description}
\item \textbf{Better support for the diversity of learners:}
\begin{description}
\item[Accessible:] current MOOC platforms are difficult to use by deaf, blind, and other disabled learners. More research is needed to understand how to make them accessible.
\item[Usable by to underserved learners:] MOOCs are sometimes thought as a way to drive illiterate adults and droppers back to school. But adequate strategies, interfaces, etc. are needed for these specific learners.
\item[Aware of cultural diversity:] offering courses in different languages is essential for offering education to everyone. Besides the diversity of languages, more research is needed to present the courses and the exercises in a way that is compatible with the different (academic) cultures.


  % \item[??diversity:] not all courses are organized around the talk of a professor. More research is needed to understand how to make MOOCs more diverse, by integrating multi-modal user interfaces, remote labs, etc. in order to implement active pedagogy in MOOCs.
  % \item[Enhancing pedagogy] help teachers to improve quality of MOOCs.
\end{description}
\end{itemize}

 The history of each science can be divided into two eras: the first when the scientists do not use any measuring instrument and the second where they do. For instance, astronomy was first naked eye astronomy and then has been instrumented since Galileo's telescope; biology has been instrumented since van Leeuwenhoek's microscope, but mathematics have been instrumented only since the middle of seventies (of the 20th century) and the use of a computer to prove the four color theorem. If the computer is the mathematician's telescope, the network may be the social scientist's telescope, as it permits to observe  social interaction on a large scale. This is particularly the case with pedagogy: MOOCs permit to observe at large scale how people teach, how people learn, etc. Thus MOOC platforms can serve as a novel measuring instrument in pedagogy. On the other hand, access to detailed MOOCs data traces requires respecting the personal rights of learners and teachers.  Research is needed for evaluating the quality of a MOOC and measuring the activities of learners in an ethical way:

\begin{itemize}
\item \textbf{Better analytics of course data:}
\begin{description}
\item[Evaluating quality:] more research is needed to evaluate the quality of a MOOC. Besides the number of learners, the drop-out rate, and the reputation of the course, factors such as learning effectiveness, fulfilment of learning objectives, or the  quality of the student cohort have to be investigated.
\item[Measuring learner activities:] developing instrumented pedagogy requires improving  learning analytics; e.g. analyzing the data traces of the learners in order to understand and predict the  progress of learners or to help teachers in planning supporting interventions and in improving the content of a course.
\item[Ethical:] more research on the ethical use of MOOCs data is needed to understand which kind of information should be accessible to whom, focusing on the privacy issues related to the data produced by the learners. Besides defining goals, research is also needed to understand their implementation in MOOC platforms, requiring, for instance, to develop privacy models and methods to develop MOOC platforms in such a way privacy is enforced by design.
\end{description}
\end{itemize}

Finally, more research is needed to understand the societal and economic impact of MOOCs. Some interesting research questions are:  how is it possible to transform on-line communities into geographically confined ones, what are the different business models to develop MOOCs and MOOC platforms, e.g.\ how to balance the efforts of private industry and public services.

%%% Local Variables: 
%%% mode: latex
%%% TeX-master: "manifesto"
%%% End: 
