State of the art

Massive Open Online Courses (MOOCs) have emerged in a period of only a
few years from 2008 into a much-discussed component of higher education
worldwide, despite a very small scale of educational provision in
comparison to traditional university education (~700 short MOOCs at the
start of 2014 versus 100,000s of degrees).  Their nature distinguishes
them from mainstream university education: they are entirely online
courses which are fully open and free of charge to anyone with an
internet connection, with large enrolments (10,000s to 100,000s of
learners) but very few teachers/tutors, and studied almost entirely
without the option of formal university credits being awarded.

We shall not review MOOCs in great detail here, as much has been written
about them in numerous recent reports (\textbf{TBD: biblio somewhere---
to include EUA, CRA, UUK, EU etc}), but shall summarise the state of
the art as of March 2014 when we met at Dagstuhl.
We shall mainly focus on university-provided MOOCs and not consider
the growing number of MOOCs from corporates, governmental and
non-governmental agencies and organisations, and from schools/VET
colleges.

\textbf{TBD: what are VET colleges?  define for non-European readers}

MOOCs take more than one form.  At one end of the spectrum is the xMOOC
which is characterised by a rather tight structure, little social
interaction and mainly computer-marked assessments, and at the other end
is the cMOOC which is almost entirely free of pre-provided content but
is characterised by very high social interactivity which produces the
content and the outcomes of the course.  Most MOOCs at present lie
between these extremes, with some structure (weekly content in the form
of video and quizzes) and with some important social interactions
(discussions, peer-review of work etc).

Although the ``year of the MOOC'', 2012, began the intense interest in
MOOCs with the launch of xMOOCs from US university computer science
departments, at present there are more MOOCs in humanities and social
science subjects than in sciences, and weekly new MOOCs appear in new
subjects.  One relative constant feature has been their focus on ``end of
high school---first year of university,'' with few MOOCs offered above or
below this level.  Various companies and organisations have been created
to act as   ``platforms'' for MOOCs, and few universities host their own
MOOCs. These platforms are commercial for profit (eg Coursera, USA),
commercial not-for-profit (eg FutureLearn, UK), noncommercial
not-for-profit (edX, USA), and governmental (eg FUN, France).

The large scale of enrolments on MOOCs (very popular MOOCs can attract
largely more that 100,000 initial enrolments) has led to intense press
and political interest in MOOCs, but the lack of retention of the great
majority of those enrollees through the whole MOOC course has led to
criticism about drop-out levels and MOOC quality.  Data about those who
enrol shows them to come from all countries, but mainly from the
educated adult population, contradicting for the present early
expectations of reaching the disadvantaged or under-represented in
higher education.  Despite this overall preponderance of an educated
learner base, MOOCs clearly do reach some learners for whom they
represent an important educational opportunity, e.g. in remote regions,
with various disabilities.

The large scale of enrolments and engagement with the online learning
materials and activities generates large amounts of data about the
learners' behaviour online.  The availability of such data has driven
great interest in the educational research community in learning
analytics, and the possibility of helping learners through better
understanding of how they are learning and what challenges they are
facing.  However alongside this positive view of the data is a ``darker
side'' of concerns about privacy of learner behaviour online and the
ethics of research and data-sharing.  The presence of commercial
companies, funded by venture capital, increases this sense of discomfort
for some in the academic and learner communities.

A result of tackling difficult questions about how to support learning
on a large scale has also led to a steady pace of innovations in
technology solutions.  Tools to manage small groups in a large
enrolment, for large-scale peer review and assessment, for visualising
analytics of a course and of individuals, mechanism to feedback progress
to learners have all emerged in the past two years.

The co-presence of MOOCs and traditional teaching in universities has
inevitably led to cross-fertilisation, and in particular to explorations
of how to use MOOCs with students enrolled on degree programmes in
particular.  Examples here are ``MOOCs as books'', Flipped or Inverted
Classrooms, Small Private Online Courses (SPOCs) based on open MOOCs,
and credit for students enrolled in external MOOCs.

The group which met at Dagstuhl came together with all these
developments and more in mind, to discuss trends in MOOCs and to share
their experiences from their own universities.  Specifically we wished
to brainstorm the research questions that have emerged and are emerging,
in order to create an agenda for research with a strong focus on the
role of computer science, artificial intelligence and learning sciences
in helping develop better MOOCs for wider audiences and more effective
learning.

\textbf{TBD:}  Armando: I think much of what's in the Manifesto also speaks to
teachers, learners, policy makers, and leaders in higher education.  The
above paragraph makes it sound like those people cannot expect much from
the report.  I think summarizing the 8 areas here, and making clear that
we are targeting all those stakeholders, would be a good idea.

To address all these stakeholders, we created the set of questions and
research topics that are set out in the following sections.

\textbf{TBD: list sections here in correct order, as map of report}
