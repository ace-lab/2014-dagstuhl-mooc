\section{State of the art}

Massive Open Online Courses (MOOCs) have emerged in a period of only a
few years from 2008 into a much-discussed component of higher education
worldwide, despite a very small scale of educational provision in
comparison to traditional university education (\textasciitilde{}700
short MOOCs at the 
start of 2014 versus 100.000s of degrees).  Their nature distinguishes
them from mainstream university education: they are entirely online
courses which are open for enrolment regardless of previous educational attainment and free of charge to anyone with an
internet connection, with large enrolments (10.000s to 100.000s of
learners\footnote{Note that in all the Manifesto, we name MOOC participants as 'learners' and retain the word 'student' for enrolled students in universities.}) but very few teachers/tutors, and studied almost entirely
without the option of formal university credits being awarded.

We do not review MOOCs in great detail here, as much has been written
about them in numerous recent reports and recommendations 
\cite{gaebel-moocs-2013,uk.gov.mooc-2013,UUK-mooc-2013,
 past-2013,InvasionoftheMOOCs-2014,mroe-2013-report,
 moocs-expectations-and-reality},
but instead summarise the state of
the art as of March 2014 when we met at Dagstuhl.
We focus mainly on university-provided MOOCs rather than
the growing number of MOOCs from corporates, governmental and
non-governmental agencies and organisations, and from schools or VET
(vocational education and training)
colleges (e.g.\ \url{http://vetcollege.in}).

MOOCs take multiple forms.  At one end of the spectrum is the xMOOC,
which is characterised by a rather tight structure, little social
interaction and mainly computer-marked assessments.  
At the other end
is the cMOOC or Connectionist MOOC, which is almost entirely free of
pre-provided content and relies instead on very high social
interactivity to produce the 
course content and outcomes.  Most current MOOCs lie
between these extremes, with some structure (weekly content in the form
of video and quizzes) and some important social interactions
(discussions, peer-review of work, and so on).

Although the ``year of the MOOC'' (2012) launched intense interest in
MOOCs with the launch of xMOOCs from computer science
departments at US universities, at present there are more MOOCs in
humanities and social 
science subjects than in the sciences, and new MOOCs appear weekly in new
subjects.  One relatively constant feature has been the focus on ``end of
high school---first year of university,'' with few MOOCs offered above or
below this level.  Various companies and organisations have been created
to act as   ``platforms'' for MOOCs, and few universities host their own
MOOCs. These platforms include commercial for-profit (Coursera, USA),
commercial not-for-profit (FutureLearn\footnote{FutureLearn is formally a commercial 
for-profit organisation but wholly owned by a not-for-profit University, and therefore 
any profits will be put back into the organisation.}, UK), noncommercial
not-for-profit (edX, USA), and governmental (FUN, France).

Very popular MOOCs can attract
more than 100.000 initial enrolments.  This scale has led to intense press
and political interest in MOOCs, but the lack of retention of the great
majority of those enrollees has led to
criticism about drop-out levels and MOOC quality.  
DeBoer and others have argued, however, that the concept of ``retention''
should be reconceptualized for MOOCs~\cite{deboer-ho-reconceptualizing},
given the very different 
risk/benefit profile that MOOCs offer relative to traditional
credit-bearing courses
that charge a fee or tuition.
Data shows that enrollees come from all countries, but mainly from the
educated adult population, contradicting (for the present) early
expectations of reaching the disadvantaged or under-represented in
higher education.  Despite this overall preponderance of an educated
learner base, MOOCs clearly do reach some learners for whom they
represent an important educational opportunity, such as learners in
remote regions or
with various disabilities.

The large scale of enrolments and engagement with the online learning
materials and activities generates large amounts of data about the
learners' behaviour online.  The availability of such data has driven
great interest in the educational research community in learning
analytics, and the possibility of helping learners through better
understanding of how they are learning and what challenges they are
facing.  However alongside this positive view of the data is a ``darker
side'' of concerns about privacy of learner behaviour online and the
ethics of research and data-sharing.  The presence of commercial
companies, funded by venture capital, increases this sense of discomfort
for some in the academic and learner communities.

Tackling difficult questions about supporting learning
at large scale has spurred a steady pace of innovation in
technology solutions.  Tools for managing small groups in a large
cohort, for large-scale peer review and assessment, for visualising
analytics of a course and of individuals, and for providing targeted
feedback 
to learners have all emerged in the past two years.  
%% \\
%% \tbd{Do we need to   cite specific ones, or would that be favoritism?  We all have our
%%   ``favorite'' work in these areas.}
%% \\tbd{CK: in my opinion, leave it like that, no further citation needed}

The co-presence of MOOCs and traditional teaching in universities has
inevitably led to cross-fertilisation, and in particular to explorations
of how to use MOOCs with students enrolled on degree programmes in
particular.  Examples include ``MOOCs as books'', flipped or inverted
classrooms, Small Private Online Courses (SPOCs) based on open MOOCs,
and credit for students enrolled in external MOOCs.

The group that met at Dagstuhl came together with all these
developments and more in mind, to discuss trends in MOOCs and to share
their experiences from their own universities.  
We wished to inform not only researchers and instructors, but also
learners, policy makers, and leaders in higher education.  The first day
of the workshop was spent brainstorming a set of questions and topic
areas based on position statements provided by each participant.
Subsequent days were spent discussing these topics, summarizing the
discussions and presenting them to the group, and drafting
recommendations based on these discussions.
Our hope is to 
%% Specifically we wished
%% to brainstorm the research questions that have emerged and are emerging,
%% in order to 
influence the creation of an agenda that will not only improve the
quality and availability of MOOCs,
but also address fundamental questions about the relationship between
MOOCs and traditional education from the perspectives of pedagogy,
administration, cost, and learner privacy.
Given the composition of our group, our discussion has a strong focus on the
roles of computer science, artificial intelligence, and the learning sciences.


%%% Local Variables: 
%%% mode: latex
%%% TeX-master: "manifesto"
%%% End: 
