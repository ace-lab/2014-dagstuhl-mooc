\section{Standards for participation metrics are necessary}

%% section editor: Pierre Dillenbourg (revised by Pierre - 22 mars)

When governments, university administrators, instructors, students, and
the general public consider a MOOC, measures of participation are
critical in their perceptions and attitudes about the MOOC's educational
value. For example, when participation is measured in terms of number of
videos watched and assessments taken, there appears to be large
attrition over time with virtually all MOOCs to date, regardless of form
or initial enrollment. Interpreted na\"{i}vely, these observations can be
taken as a negative aspect of MOOCs, but such measures fail to consider
the goals of the learner and other factors that might distinguish
subpopulations with very different participation patterns.
Participation measures will inform assessments of (a) individual student
learning outcomes, which is perhaps primary. But through these
assessments, participation measures will also inform assessment of (b)
cohorts of students; (c) resources, the utility of which depends in part
on use of these resources, and the student outcomes that result; (d)
instructors, whose quality depends on the outcomes of courses they
teach; and (e) the quality of the MOOCs themselves and higher-level
administrative and educational structures.  As yet, we know of no
consensus, detailed inventory of variables for measuring participation
in MOOCs, which could guide pedagogy, practice, and policy.
There is
important research to be done on defining measures of participation and
other educational variables for MOOCs~\cite{deboer-ho-reconceptualizing},
studying the effect of these measures on the perceptions of different
stakeholders, and on how best to obtain and protect the
learner data needed for highly nuanced, conditionalized
measures of participation.  

\subsection{Building blocks for deriving participation measures}

A starting point for a more comprehensive picture of
participation is measuring students' actions regarding: 

\begin{itemize}

\item MOOC resources,
such as videos, assessments, readings and so on;

\item assessments, both
formative and summative, both attempted and completed; 

\item resource
annotation, to include tagging video and textual sources; 

\item resource
creation (common in connectionist MOOCs), such as video and text; and

\item communications with the instructors, assistants, MOOC platform company,
and with other students.

\end{itemize}

These categories are not mutually exclusive:  a discussion board
post in response to another student post is both a communication action
and a resource annotation, where the initial post
is a resource.  Neither are these categories
complete. Finally, each category includes many
sub-categories, as with  formative vs.\ summative
assessment. A rich assessment inventory will almost certainly be
hierarchically structured, with many categories and sub-categories, and
with fine grained measures possible, if not desirable.

Importantly, within this rich set of variables will be \emph{base variables},
which are nonnormalized and often continuous (for example, how many videos
watched, total quiz score); and interpreted variables, which are often
discrete, even binary, and convey value or judgement about a
participant's activity in a MOOC (for example, pass/fail). Here, we
are most interested in the possibilities for participation within a
multidimensional space of base variables, with the understanding that
interpreted variables can be introduced with thresholds that are
particular to differing conditions, preferences, constraints, and goals.

While independent measurements of and correlational analysis between
different variables 
enable important statistical relationships to be found---for example,
correlating number of video hours watched with final course score---more causal
interpretations can be aided by capturing \emph{traces} of student
activities. A trace is a sequential or tree-based representation of a
student's activities, and enables analysis beyond pairwise associations
(e.g., ``\texttt{user1} read a post by \texttt{user2} directing them to
\texttt{resource25},  and
\texttt{user1} followed the recommendation''). By comparing student traces, for
example through graph clustering or inductive logic programming, more
complicated patterns of activities can be found across a population of
students. While recording a student's behavior as a vector of variable
values is a popular representation in data mining generally, traces and
other higher-order representations need to be developed and explored
further, for example as in~\cite{activity-tracing}.  

\subsection{The wealth of conditioning information}

Participation patterns
undoubtedly vary with many factors, and we should strive to capture, and
explain variance on participation variables based on these many
factors. Let us briefly consider the MOOC design, attributes
of individual students and student cohorts, and time, as major
categories of independent variables that will condition findings on MOOC
participation. An important insight presented in~\cite{mroe-2013-report}
and elsewhere is that participation is
a value or region in a rich and continuous multidimensional space.

\textbf{MOOC Designs.} Even today, there are several recognized kinds of MOOCs
(e.g., xMOOC vs iMOOC vs cMOOC), to say nothing of the MOOC types that
will emerge in the coming years.  Rather than standardizing measures
across diverse MOOC formats, harmonisation of measures that are best
suited to the different types of MOOCs seems the better view. For
example, resource creation by students will be relevant measures of
participation to some MOOCs. Some MOOCs will be inherently more social,
with requirements for collaboration, whereas others will not. Some
measure simply will not apply to every type of MOOC. Thus, measures will
apply differentially across MOOCs, and garnering a comprehensive set of
measures across current and future MOOC formats will be a challenge. We
recommend a catalog of such measures, perhaps in wiki form, which can be
open to the research and practitioner communities and easily linked to
other sources.  In addition to cataloging measures that may vary with
format, there will hopefully be ``canonical'' measures that are
independent of MOOC types, with the simplest of these being ``time spent''
by the student on the MOOC (e.g., watching its videos, taking
assessments, doing projects). While this is a simple measure to state,
it may be very challenging to capture, because much of the time spent on
a MOOC probably happens off the MOOC platform.  

\textbf{Student characteristics.} Student characteristics influence
participation. Primary among these characteristics are the motivations
that the student has for taking a MOOC -- does the student intend to
complete, or is the student there to sample the content only? Related to
completion versus sampling goals, but distinct, is the level of mastery
that a student hopes to achieve. If a student wishes to complete a
course, can we capture information on whether a student returns to the
MOOC after ``dropping'' it previously?  In addition to being critical for
a descriptive understanding of student participation under many nuanced
conditions, a rich characterization of students will inform
prescriptions (e.g., suggestions) of how students might best work their
way through MOOC material; who might be the best partners, peer
reviewers, mentors, and instructors for students in different contexts;
thus influencing participation patterns actively through recommender and
other help systems. In other words, the many affordances that we want to
create for students will drive the conditioning variables that will be
measured.  

\textbf{Time and other continuous dimensions.}  Although participation
at a single point in time is of value, it is the detailed and real-time
participation information that will be most relevant to both the
educator and student. This data can be used to improve the learner
experience and inform MOOC design making it more responsive and
adaptive. This is especially the case in identifying different user
types amongst MOOC students and potentially factoring this into the
overall way the MOOC adapts to the different needs of individual
students and the groups they belong to. Thus, for each of the dependent
and conditioning variables that we alluded to above, we will want
measures over time, annotated by events that have predictive value.
Attention to the time dimension also invites analysis using traces,
which was mentioned above. For example, student actions at one time
point influence actions and outcomes at subsequent time points, all of
which can be represented, processed, and compared by computational
means.  Time is a primary, but certainly not the only, continuous
dimension for which various measures of participation will be continuous
functions rather than simply point values!

\subsection{Challenges in assessing and improving participation}

 There are many challenges in defining participation, in
obtaining data on participation, assessing it, reporting it, and
eventually comparing participation analyses across MOOCs and
MOOC-hosting platforms. A main challenge, described in the previous
paragraphs, is the diversity of MOOCs and the diversity of types of
participants. Another obvious challenge are privacy concerns that will
stem from heavy instrumentation of MOOCs and MOOC platforms to collect
data. The measurement of student activity may become intrusive for
students concerned. What happens to participation measures if students
are required to agree to be measured? How do we safeguard unintended
privacy leaks through the use of detail at this level?  Despite these
challenges, we believe important for our community to develop
participation metrics. An understanding of MOOC participation will bring
a greater understanding of the impacts of external factors on learning
in a MOOC: the subject matter of the MOOC, its duration and pace,
lecturer, institutions and sponsors behind the MOOC, fees if any,
transparency in students' data collection and use, and the availability
of credits or certificates for completion.

