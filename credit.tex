\section{Universities should provide official credits for MOOCs}
\label{sec:credit}
 
\subsection{Why credits?}

In a nutshell, credits are the conditoons to create an European or
worldwide educational landscape. 
Currently, most universities do not provide official credits for MOOCs,
i.e. the student receives a certificate that cannot be counted as a
subset of an on-campus degree. This is especially important in Europe where
ceredits are formally defined in terms of students workload and serve as
an exchange currency. Thereby, credits allow mobility between curricula inside
institutions and between institutions. If MOOCs would lead to official
credits, this current physical mobility schemes would be enhanced with 
virtual mobility. In turn, this would allow the construction of curricula
across institutions. Building cross-instituions curricula is already a
pratice with many instituional, technical and logistics constraints. 
Technical constraints raise the issue of interoperability if MOOCs are not
running on same platform. Credits wave some of the organisation constraints
in building distributed curricula. We hence prooise to offer  credits for 
MOOC achievement as soon as the two following conditions are met: (1) the identity of the
learner has been verified and (2) there exists a valid
measure of the target skills. Identity verification in an online context
is an intense field of development, where we will soon reach appropriate
solutions. The validity of the assessment is in the responsibility of
the teacher and belongs to the quality control that universities and
platforms are expected to operate anyhow.

In addition, offering credits has potentially deep consequences on the MOOC
ecosystem, namely on the high attrition rate of MOOCs. Actually, the
current drop-out rates are overestimated. Many registered students never
show up (sometimes up to 30\%). In addition, some participants only
register in order to be able to access the content provided by the MOOC
but without the intention to take the full course. Despite these two
elements, the rate of drop-outs remains high, which may interfere with a
credit system. In a nutshell, credits increase both the responsibilities
of learners and teachers. Students may feel more motivated to complete a
MOOC if the certificate is useful for the rest of their
studies. Universities may feel more concerned in bringing students to
achievement if the credits are necessary for a longer educational
process.  To reach a more reasonable completion rate, there are two
possibilities: (a) to carefully select the participants based on their
levels of pre-requisites and (b) to better support them during the MOOC,
especially those facing difficulties. 


\subsection{MOOCs and Prerequisites}

MOOCs raise a debate between effectiveness and openness. There are two
views: (a) One meaning of  ``open''  is that MOOCs are accessible to any
student independently of his background. (b) If those who do not have the prerequisites were filtered
out early on, the achievement rate would be higher. This seems to be especially
an issue if failing student occurs to be an alumni from the institution that provides the
MOOC: his or her frustration may be detrimental to
universities that are substantially funded by their alumni.

This is a general issue in academia which runs on a combination of training and selection: we are willing to
accept more students in our universities but afraid that this could lower the level.
In this a/b debate, there will be a diversity of approaches, closer to a or to be, 
as there are already universities in different positions on this continuum. However,
we bend on the side a for the following reasons. 
First, it occurs that students without prerequisites nonetheless succeed, due to higher
engagement. Second,  admission criteria may be somewhat
imprecise and based on limited information. Hence, it is appealing to allow
everyone the chance to try for success. Therefore, we propose to favor option a against option b: 
we believe students should have the right to take the risk of joining a MOOC despite missing
prerequisites. This hsould however be accompagnied by initiatives by which more students
could achieve a MOOC and obtain credits, while keeping both the current
policy of universal access and the level of expectation on final outcomes. 
These initiatives could explore richer
individual support, social dynamics (e.g. meet-up groups), analytics for
drop-out prediction, increase time flexibility, etc. In general, we
hypothesize that a strong investment in the quality of teaching will
contribute to lower attrition; we want students to feel that teachers
help them with hard process.

The rate of achievement is becoming an even more important concern with
credits when credits are  aggregated in order to get degrees:
if 50\% of the students succeed the first MOOC, and among them 50\%
succeed the second MOOC, only 25\% of the original cohort are still on
the path towards a degree. Moreover, a degree usually covers
higher-order learning outcomes (creativity, sense of rigour, critical
analytic skills, skills of synthesis, reflection, ability to identify
problems, social skills, and so on) which are rarely addressed in MOOCs. We
recommend research on MOOC activities that support the development of
these high-level skills. Replacing exams by projects or even capstone
projects are examples of such activities.


\subsection{Universities' Role in Credit-Articulation Systems}

Will  universities retain their monopoly
of issuing credits and certificates, or will a broader variety of actors,
such as professional or accreditation organisations, 
also decide to give credits?  

For example, in Europe, the European Credit Transfer and Accumulation
(ECTS) system could regulate the
development of the campus$+$MOOC academic landscape.  In the US, the
California Community Colleges (CCC) system, which includes over 100 campuses
serving 2.4 million students, has ``articulation agreements'' that allow
transfer of specific course credits to degree programs at other institutions,
including University of~California campuses, various independent public
and private colleges in California, and even Arizona State University.
If the accreditation ecosystem evolves in a natural
way, the value of MOOC credits will depend upon the reputation of the
institution that delivers them.

We expect that universities may take an
incremental approach, initially offering
credit for separate versions of MOOC material that serve a large, but
not necessarily massive or open, population.
For example, students who sign up for ``verified''
enrollment, such as Coursera's Signature Track or edX Verified
Certificates, might be put
into separate discussion groups from the others.  
What obstacles, if any, currently exist to prevent such a scenario, and
should universities pursue such scenarios?
Reasons not to do so could include decreasing the educational value of
residential credit, interfering with the schools' business model, or the
need for a proctored follow-on activity on campus to
weed out individuals who had cheated egregiously in the MOOC.  On the
positive side, such an activity would also be a way to build affinity to
the brick-and-mortar university, which could be a net win for both the
student and the university.
 


