\section{Universities should provide official credits for MOOCs}
\label{sec:credit}
 
% PCAST letter on MOOCS - 3 recommendations, one of which is
% ``Encourage accrediting bodies to be flexible in response to edu
% innovation''
% (http://whitehouse.gov/sites/default/files/microsites/ostp/PCAST/pcast_edit_dec-2013.pdf)

\subsection{Why credits?}

In a nutshell, credits are a necessary condition for MOOCs to have a significant impact on  the European or
worldwide educational landscape. 

Currently, most universities do not provide official credits towards an
on-campus degree for MOOC
completion certificates.
This is especially important in Europe where
credits are formally defined in terms of learners workload and serve as
an exchange currency, thereby allowing mobility among curricula within an
institution and among institutions. If MOOCs were recognized with official
credits, this current physical mobility would be enhanced with 
virtual mobility.  This would complement parallel developments towards virtual mobility in taught online courses.
Physical plus virtual mobility would both allow and motivate
the construction of cross-institutional curricula, something that
already occurs in practice but is hampered by
many institutional, technical and logistical constraints. 


Technical constraints include the issue of interoperability in particular since MOOCs are 
running on different platforms. 
   % ck: Credits wave some of the organisation constraints
   % in building distributed curricula. \tbd{What??} 
We hence propose that credits should be offered 
by universities for MOOC achievement if the two following conditions are met: (1) the identity of the
learner has been verified and (2) there exists a valid
measure of the target knowledge and skills. Identity verification in an online context
is an area of intense development in which appropriate solutions are
starting to emerge. The validity of the assessment is the responsibility of
the teacher and is part of the quality control that universities and
platforms are expected to exercise anyhow.

Offering credit has potentially deep consequences on the MOOC
ecosystem, for example on the high attrition rate of MOOCs. 
\tbd{ck: comment of Francois Bry: In Germany and Austria, a similar rate of students never showing
  up is observed in face-to-face higer education - apparently, acros universities and fields.  This
  reflects a novel attitude towards education which is not specific to MOOCs. }
Actually, the
current drop-out rates are overestimated: up to 30\% of registered learners never
even show up. In addition, some participants only
register in order to be able to access the content provided by the MOOC
but with no intention of taking the full course. Even taking into account these two elements, 
the rate of drop-out remains high, which may interfere with a
credit system. Credit can help because it increases the responsibilities
of both learners and teachers: Learners may feel more motivated to complete a
MOOC if the certificate is useful for the rest of their
studies, and universities may feel more concerned in bringing learners to
a designated level of achievement if the credits are necessary for a longer educational
process.  To reach a more reasonable completion rate, we can pursue two strategies:
(a)  carefully select participants based on their
mastery of prerequisites, and (b)  better support them during the MOOC,
especially those facing difficulties. 


\subsection{MOOCs and Prerequisites}

MOOCs highlight a tension between effectiveness and openness. 
On the one hand, one meaning of  ``open''  is that MOOCs are accessible to any
person independently of her or his background.  On the other hand,
if those lacking the proper prerequisites were filtered
out early on, the MOOC achievement rate might be higher. This might
especially be
an issue if the failing learner happens to be an alumnus of the institution providing the
MOOC: his or her frustration may be detrimental to
university, and many universities are substantially funded by their alumni. 

\tbd{ck: 
comment by fb: " many universities are substantially funded by their alumni." Not in
  continental Europe! I dare say that that in continental European universities there is not much
  culture of caring whether students (asnd for that matter teachers) might be frustrated. 
comment by jh: I am unsure if there is evidence of drop-out due to MOOCs being too difficult.  
We didnt see this, it was time and interest that dominated drop-out.  indeed drop-outs often dont reply to surveys.....
}

This is a general issue in academia, which runs on a combination of training and selection: we are willing to
accept more learners in our universities but afraid that this could
lower overall excellence or learner achievement.
Different universities already take different positions on this continuum. However,
for MOOCs we favor the former option---accepting more learners---for the following reasons. 
First,  learners without prerequisites nonetheless do succeed sometimes, due to higher
engagement. Second,  admission criteria may be somewhat
imprecise and based on limited information. Hence, it is appealing to allow
everyone the chance to try for success. Therefore, we suggest
that learners should have the right to take the risk of enrolling in a MOOC despite lacking
some prerequisites. 

\tbd{ck: comment by fb: In my humble opinion, the paragraph starting with "This openness should…"
  could be refined so as not to look too naive.

  A MOOC making it pissible that an educational team of n persons accomodate well 100.000 learners
  is likely to colapse in presence of 120.000 learners.

  Is it wise to propagate in this maifesto the view that a MOOC software makes it possible to teach
  to an unbounded number of studsents with a bouned number of techers?

  In fact, traditional education scales exceelently. Indeed, it scales lineraly. The point is, that
  MOOCs seem for the first timer in human history to make a subinear scaling possible in higher
  education. A key issue is how to achieve this. For sure, this will not be achieved by pretending
  to have a bounded number of teachers educate an anbounded number of students!

  Thus, this section on MOOCS' Current State and Perspectives" should end stressing that essential
  research perspectiveare

  - to design teaching approaches and the social software they need for achieving the afore-mentoned
  sublinear scaling,

  - collect learning analytcs from MOOCs so as to realize a feedback loop so as to elarn how to
  achive a better scaling. }

This openness should, however, be accompanied by initiatives through which more learners
could achieve a MOOC and obtain credits, while keeping both the current
policy of universal access and the level of expectation on final outcomes. 
These initiatives could explore richer
individual support, social dynamics such as meet-up groups, analytics for
drop-out prediction, increased time flexibility, and so on. In general, we
hypothesize that a strong investment in the quality of teaching will
contribute to lower attrition; we want learners to feel that teachers
are there to help them navigate difficult learning processes. 

The rate of achievement is an even more important concern with
credits when credits are  accumulated in order to get degrees:
if 50\% of the learners pass the first MOOC, and among them 50\%
pass the second MOOC, only 25\% of the original cohort are still on
track towards a degree. Moreover, a degree usually covers
higher-order learning outcomes that are rarely addressed in MOOCs, such
as creativity, sense of rigour, critical
analytic skills, skills of synthesis, reflection, ability to identify
problems, social skills, and so on. We
recommend research on MOOC activities that support the development of
these high-level skills. Replacing exams by projects or even capstone
projects are examples of such activities.


\subsection{Universities' Role in Credit-Articulation Systems}

Will universities retain their relative monopoly
of issuing credits and certificates for advanced study, or will a broader variety of actors,
such as professional or accreditation organisations, 
also decide to give credits?
\tbd{comment by jh: i feel that this sentence is too simplistic and suggests that only universities
  give credit and awards for advanced study. Even with degrees this varies between countries.  Most
  importantly, in advanced professional education professional bodies often offer training and
  credits needed for licence to practice (law, engineering, medicine etc..).
  %
  i am not sure at the moment how best to re-phrase!!
 ck: the sentence is ok with me as it is and the concern of Jeff is somehow treated in the rest of
 the section
}
For example, in Europe, the European Credit Transfer and Accumulation
System (ECTS) could regulate the
development of the campus$+$MOOC academic landscape.  Some open
universities such as iVersity already offer ECTS credits for some MOOCs
(\url{https://iversity.org/pages/moocs-for-credit }). 
% UNED Abierta (Univ Nacional de Educacion a Distancia, Spain) awards
% both ``verified participation'' type certs and ECTS-accredited certs
% for some MOOCs(COMA - curso online masivo y abierto), but I can't find
% the ref on portal.uned.es
% EduKart offers a variety of online degrees accredited by the Indian
% University Grant Commission (UGC): edukart.com
In the USA, the
American Council on Education has recommended a number of
%% now 11??
MOOCs for undergraduate credit, although each university ultimately
decides which transfer credits it is willing to accept.
%% but who accepts the credit??
The undergraduate-credit MOOCs are
offered by Coursera, whose ``Signature Track'' option gives learners the
option to  undergo  a series of identity-verification measures and take
a proctored online exam after the course ends.
%% California Community Colleges (CCC) system, which includes over 100 campuses
%% serving 2.4 million students, has ``articulation agreements'' that allow
%% transfer of specific course credits to degree programs at other institutions,
%% including University of~California campuses, various independent public
%% and private colleges in California, and even Arizona State University.
If the accreditation ecosystem evolves in a natural
way, the value of MOOC credits will depend upon the reputation of the
institution that delivers them.

We expect that universities may take an
incremental approach, initially offering
credit for separate versions of MOOC material that serve a large, but
not necessarily massive or open, population.
For example, learners who sign up for ``verified''
enrollment, such as Coursera's Signature Track or edX Verified
Certificates, might be put
into separate discussion groups from the others.
Similarly, some faculty are exercising their
discretion as instructors, by allowing MOOC to informally satisfy one or
more prerequisites for courses they teach, thus enabling students to jump
more immediately into advanced courses; such bottom-up decision making may
have influence on institutional policies.
Should universities identify and tackle the obstacles to pursuing such scenarios?
Reasons not to do so could include decreasing the educational value of
residential credit, interfering with the schools' business model, or the
need for a proctored follow-on activity on campus to
weed out individuals who had cheated egregiously in the MOOC.  On the
positive side, such an activity would also be a way to build affinity to
the brick-and-mortar university, which could be a net win for both the
learner and the university.
 


