\section{Universities should provide official credits for MOOCs}
\label{sec:credit}
 
% PCAST letter on MOOCS - 3 recommendations, one of which is
% ``Encourage accrediting bodies to be flexible in response to edu
% innovation''
% (http://whitehouse.gov/sites/default/files/microsites/ostp/PCAST/pcast_edit_dec-2013.pdf)

\subsection{Why credits?}

In a nutshell, credits are necessary conditions for MOOCs to impact the European or
worldwide educational landscape. 

Currently, most universities do not provide official credits towards an
on-campus degree for MOOC
completion certificates.
This is especially important in Europe where
credits are formally defined in terms of students workload and serve as
an exchange currency, thereby allowing mobility among curricula within an
institution and among institutions. If MOOCs were recognized with official
credits, this current physical mobility would be enhanced with 
virtual mobility.  Physical plus virtual mobility would both allow and motivate
the construction of cross-institutional curricula, something that
already occurs in practice but is hampered by
many institutional, technical and logistical constraints. 


Technical constraints include the issue of interoperability if MOOCs are not
running on same platform. Credits wave some of the organisation constraints
in building distributed curricula. \tbd{What??} We hence propose to offer  credits for 
MOOC achievement as soon as the two following conditions are met: (1) the identity of the
learner has been verified and (2) there exists a valid
measure of the target skills. Identity verification in an online context
is an area of intense development in which appropriate solutions are
starting to emerge. The validity of the assessment is the responsibility of
the teacher and is part of the quality control that universities and
platforms are expected to exercise anyhow.

Offering credit has potentially deep consequences on the MOOC
ecosystem, for example on the high attrition rate of MOOCs. Actually, the
current drop-out rates are overestimated: up to 30\% of registered students never
even show up. In addition, some participants only
register in order to be able to access the content provided by the MOOC
but with no intention of taking the full course. Despite these two
elements, the rate of drop-outs remains high, which may interfere with a
credit system. Credit can help because it increases the responsibilities
of both learners and teachers: Students may feel more motivated to complete a
MOOC if the certificate is useful for the rest of their
studies, and universities may feel more concerned in bringing students to
a designated level of achievement if the credits are necessary for a longer educational
process.  To reach a more reasonable completion rate, we can pursue two strategies:
(a)  carefully select participants based on their
mastery of prerequisites, and (b)  better support them during the MOOC,
especially those facing difficulties. 


\subsection{MOOCs and Prerequisites}

MOOCs highlight a tension between effectiveness and openness. 
On the one hand, one meaning of  ``open''  is that MOOCs are accessible to any
student independently of his background.  On the other hand,
if those lacking the proper prerequisites were filtered
out early on, the MOOC achievement rate might be higher. This might
especially be
an issue if the failing student happens to be an alumnus of the institution providing the
MOOC: his or her frustration may be detrimental to
university, and many universities are substantially funded by their alumni.

This is a general issue in academia, which runs on a combination of training and selection: we are willing to
accept more students in our universities but afraid that this could
lower overall excellence or student achievement.
Different universities already take different positions on this continuum. However,
for MOOCs we favor the former option---accepting more students---for the following reasons. 
First,  students without prerequisites nonetheless do succeed sometimes, due to higher
engagement. Second,  admission criteria may be somewhat
imprecise and based on limited information. Hence, it is appealing to allow
everyone the chance to try for success. Therefore, we suggest
that students should have the right to take the risk of enrolling in a MOOC despite lacking
some prerequisites. 

This openness should, however, be accompanied by initiatives through which more students
could achieve a MOOC and obtain credits, while keeping both the current
policy of universal access and the level of expectation on final outcomes. 
These initiatives could explore richer
individual support, social dynamics such as meet-up groups, analytics for
drop-out prediction, increased time flexibility, and so on. In general, we
hypothesize that a strong investment in the quality of teaching will
contribute to lower attrition; we want students to feel that teachers
are there to help them navigate difficult learning processes. 

The rate of achievement is an even more important concern with
credits when credits are  accumulated in order to get degrees:
if 50\% of the students pass the first MOOC, and among them 50\%
pass the second MOOC, only 25\% of the original cohort are still on
track towards a degree. Moreover, a degree usually covers
higher-order learning outcomes that are rarely addressed in MOOCs, such
as creativity, sense of rigour, critical
analytic skills, skills of synthesis, reflection, ability to identify
problems, social skills, and so on. We
recommend research on MOOC activities that support the development of
these high-level skills. Replacing exams by projects or even capstone
projects are examples of such activities.


\subsection{Universities' Role in Credit-Articulation Systems}

Will  universities retain their monopoly
of issuing credits and certificates, or will a broader variety of actors,
such as professional or accreditation organisations, 
also decide to give credits?  
For example, in Europe, the European Credit Transfer and Accumulation
(ECTS) system could regulate the
development of the campus$+$MOOC academic landscape.  Some open
universities such as iVersity already offer ECTS credits for some MOOCs
(\url{https://iversity.org/pages/moocs-for-credit }). 
% UNED Abierta (Univ Nacional de Educacion a Distancia, Spain) awards
% both ``verified participation'' type certs and ECTS-accredited certs
% for some MOOCs(COMA - curso online masivo y abierto), but I can't find
% the ref on portal.uned.es
% EduKart offers a variety of online degrees accredited by the Indian
% University Grant Commission (UGC): edukart.com
In the USA, the
American Council on Education has recommended a number of
%% now 11??
MOOCs for undergraduate credit, although each university ultimately
decides which transfer credits it is willing to accept.
%% but who accepts the credit??
The undergraduate-credit MOOCs are
offered by Coursera, whose ``Signature Track'' option gives learners the
option to  undergo  a series of identity-verification measures and take
a proctored online exam after the course ends.
%% California Community Colleges (CCC) system, which includes over 100 campuses
%% serving 2.4 million students, has ``articulation agreements'' that allow
%% transfer of specific course credits to degree programs at other institutions,
%% including University of~California campuses, various independent public
%% and private colleges in California, and even Arizona State University.
If the accreditation ecosystem evolves in a natural
way, the value of MOOC credits will depend upon the reputation of the
institution that delivers them.

We expect that universities may take an
incremental approach, initially offering
credit for separate versions of MOOC material that serve a large, but
not necessarily massive or open, population.
For example, students who sign up for ``verified''
enrollment, such as Coursera's Signature Track or edX Verified
Certificates, might be put
into separate discussion groups from the others.  
Should universities identify and tackle the obstacles to pursuing such scenarios?
Reasons not to do so could include decreasing the educational value of
residential credit, interfering with the schools' business model, or the
need for a proctored follow-on activity on campus to
weed out individuals who had cheated egregiously in the MOOC.  On the
positive side, such an activity would also be a way to build affinity to
the brick-and-mortar university, which could be a net win for both the
student and the university.
 


