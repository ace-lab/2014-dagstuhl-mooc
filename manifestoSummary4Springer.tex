%This is a template for producing reports for "Dagstuhl Manifestos".
%See dagman.pdf for furhter information.

\documentclass[a4paper,UKenglish]{dagman}
  %for A4 paper format use option "a4paper", for US-letter use option "letterpaper"
  %for british hyphenation rules use option "UKenglish", for american hyphenation rules use option "USenglish"
  %for section-numbered lemmas etc., use "numberwithinsect"

\usepackage{microtype}%if unwanted, comment out or use option "draft"

%Author macros: begin%%%%%%%%%%%%%%%%%%%%%%%%%%%%%%%%%%%%%%%%%%%%%%%%%%%%%
\subject{FORUM / Manifesto from Dagstuhl Perspectives Workshop 14112}
\title{FORUM / Massive Open Online Courses: Current State and Perspectives}
%\titlerunning{MOOCs: Current State \& Perspectives}%optional

\author[1]{Pierre Dillenbourg}
  \affil[1]{EPFL, Switzerland   \texttt{pierre.dillenbourg@epfl.ch}}

\author[2]{Armando Fox}
  \affil[2]{University of California, Berkeley, USA  \texttt{fox@berkeley.edu}}

\author[3]{Claude Kirchner}
  \affil[3]{Inria, France   \texttt{claude.kirchner@inria.fr}}

\author[4]{John Mitchell}
  \affil[4]{Stanford University, USA  \texttt{john.mitchell@stanford.edu}}

\author[5]{Martin Wirsing}
  \affil[5]{Ludwig-Maximilians-Universit\"{a}t M\"{u}nchen, Germany  \texttt{wirsing@lmu.de}}

        

%% add other authors here

%% author info below should match that above:

%Dagstuhl editorial office macros: begin%%%%%%%%%%%%%%%%%%%%%%%%%%%%%%%%%%%%%
\volumeinfo%(easychair interface)
  {Pierre Dillenbourg and Armando Fox and Claude Kirchner and John Mitchell and Martin
    Wirsing }%editors
  {5}%number of editors
  {A Manifesto Sample}%event
  {1}%volume
  {1}%issue
  {1}%starting page number
%\DOI{10.4230/DagMan.1.1.1}%(DagRep.<issue no>.<volume no>.<firstpage>)
%Dagstuhl editorial office macros: end%%%%%%%%%%%%%%%%%%%%%%%%%%%%%%%%%%%%%



%% \authorrunning{J.Q. Open and J.R. Access}%optional

% mandatory: Please choose
% ACM 1998 classifications from
% http://www.acm.org/about/class/ccs98-html . E.g., cite as "F.1.1
% Models of Computation".  
\subjclass{K.3 Computers and Education}

% mandatory: Please provide 1-5 keywords
\keywords{Massively open online courses, MOOC, SPOC, e-learning, education}

\seminarnumber{14112}
\semdata{10.--13.~March, 2014 -- \href{http://www.dagstuhl.de/14112}{www.dagstuhl.de/14112}}
%% \additionaleditors{}%optional
%Author macros: end%%%%%%%%%%%%%%%%%%%%%%%%%%%%%%%%%%%%%%%%%%%%%%%%%%%%%

% macros for ``Tbd''

\newcommand{\tbd}[1]{{\colorbox{yellow}{\color{red} \textbf{TBD:} \protect{#1}}}}
%\newcommand{\tbd}[1]{}

\begin{document}

  \maketitle

%\begin{abstract}

The rapid emergence and adoption of MOOCs (Massive Open
Online Course) has raised new questions and rekindled old debates
in higher education.
Academic leaders are concerned about educational quality, access to content, 
privacy protection for learner data, production costs and the
proper relationship between MOOCs and residential instruction, among
other matters.
At the same time, these same leaders see opportunities for the scale of MOOCs
to support learning:  faculty interest in teaching innovation,
better learner engagement through personalization,
increased understanding of
learner behavior through large-scale data analytics, wider access
for continuing education learners and other nonresidential learners, and
the possibility to enhance revenue or lower educational costs.
Two years after ``the year of the MOOC'', this report summarizes the
state of the art and the future directions of greatest interest as seen
by an international group of academic leaders.
Eight provocative positions are put forward, in hopes of aiding policy makers,
academics, administrators, and learners regarding the potential future
of MOOCs in higher education.  The recommendations span a variety of
topics including financial considerations, pedagogical quality, and
the social fabric.
\end{abstract}



\begin{abstract}

  The rapid emergence and adoption of MOOCs (Massive Open Online Course) has raised new questions
  and rekindled old debates in higher education.  Academic leaders are concerned about educational
  quality, access to content, privacy protection for learner data, production costs and the proper
  relationship between MOOCs and residential instruction, among other matters.  At the same time,
  these same leaders see opportunities for the scale of MOOCs to support learning: faculty interest
  in teaching innovation, better learner engagement through personalization, increased understanding
  of learner behavior through large-scale data analytics, wider access for continuing education
  students and other nonresidential learners, and the possibility to enhance revenue or lower
  educational costs.  Two years after ``the year of the MOOC'', this workshop was an opportunity to
  summarizes the state of the art and the future directions of greatest interest as seen by an
  international group of academic leaders.  Eight provocative positions are put forward, in hopes of
  aiding policy makers, academics, administrators, and learners regarding the potential future of
  MOOCs in higher education.  The recommendations span a variety of topics including financial
  considerations, pedagogical quality, and the social fabric; they are developed
  in~\cite{DagstuhlManifesto-2014}.
\end{abstract}

%\section*{Executive Summary}
\section{Introduction}

Online education is not new; Massively Open Online Courses (MOOCs) are. 
Much has been written
about them in numerous recent reports and recommendations 
\cite{gaebel-moocs-2013,uk.gov.mooc-2013,UUK-mooc-2013,
 past-2013,InvasionoftheMOOCs-2014,mroe-2013-report,
 moocs-expectations-and-reality}.
Their uniquely powerful
combination of classical digital teaching tools (videos, audios, graphics or slides), individualized
tools for acquiring and validating knowledge, and appropriate use of dedicated social networks makes
them a new and formidable means of accessing knowledge and education. If backed up with scientific
and pedagogical excellence, MOOCs allow one to reach and teach simultaneously tens of thousands and
even hundreds of thousands of learners in a new pedagogical dynamic.

The computer science and informatics community has long collaborated with the teaching and knowledge
dissemination community through the creation and deployment of technology for teaching and learning.
However, MOOCs seem to represent a new level of engagement between these communities because of
their scale, their links to economic and production systems in higher education, and the
conversations about teaching that they have provoked, some of which may induce radical changes in
teaching mechanisms. The consequences on transmission of culture and educational content, and on
society as a whole, will be deep.

The Perspectives Workshop on ``Massively Open Online Courses, Current State and Perspectives'' took
place at Schloss Dagstuhl on March 10--13, 2014.  Twenty-three leading researchers and practitioners
from informatics and pedagogical sciences presented and discussed current experiences and future
directions, challenges, and visions for the influence of MOOCs on university teaching and learning.
The first day of the workshop consisted of a series of presentations in which each participant
presented those topics and developments he or she considered most relevant for the future
development of MOOCs. On the second and third day the participants divided into several working
groups according to the main thematic areas that had been identified on the first day.

This manifesto summarizes the key findings of the workshop and provides a collection of research
topics for MOOCs. The eight theses presented in the report can be divided into three categories: the
integration of MOOCs into university education, quality assurance and measures of success, and
policies for access and privacy.

The working group chose to make specific recommendations that are intended provoke discussion and
lead to better understanding of the core issues. The manifesto represents the editors' best effort
to capture the main positions put forward during the meeting, but may vary from the opinions of
individual participants.  This is a summary of the full manifesto to appear in the Dagstuhl
Perspectives Workshop serie~\cite{DagstuhlManifesto-2014}.


\section{Integration of MOOCs into university education}

\subsubsection*{Universities should refactor residential education using MOOCs}

The MOOC-accelerated paradigm shift in education, comparable to shifts in the news media and the
music industry, is forcing an unbundling and rebundling of educational components. In the case of
higher education, we must rethink long-established modes of interaction, teaching roles, and
structures in a way that recognizes that the sources of knowledge and therefore the roles of
instructors and campuses have changed.

We see the most important shifts as follows:

\begin{itemize}

\item With the wide availability of high-quality, (often) free, and easy-to-reuse content online,
  most instructors' primary responsibility shifts \emph{from creating content to creating context}.

\item Rearrangements of residential course elements (lectures, recitations, labs, and so on) and new
  teaching roles beyond simply ``instructor'' and ``teaching assistant'' will better fit this new
  instructional modality.  One challenge of this reorganization is changes to campus infrastructure,
  but one facilitator is that inexpensive video production will empower domain experts to create
  better materials to prepare and train the new teaching roles.

\item The absence of the ``sage on the stage'' will open new ways to foster teacher and learner
  commitment.  Campuses should focus less on conveying content-oriented skills and more on
  social/professional skills, such as collaborative work and perspective-broadening activities, to
  complement independent study and discovery.

\end{itemize} 

\subsubsection*{New pedagogical approaches should be developed to take advantage of scale}

MOOCs offer new opportunities for scaling up education to very large numbers of learners: they can
reach up to 200.000 learners, compared with the approximately 500 in a typical lecture or seminar
room.  This scale is made possible in part by the ``open'' features of MOOCs and their worldwide
accessibility and leads to the two following opportunities.

\begin{description}
\item[Rich despite scale] We care about pedagogical diversity at scale and this leads us to the
  following recommendation expressed as a research question: How can we create a broad range of
  effective pedagogies at massive scale, and efficiently contribute to achieve 21st century
  competencies? The challenge is to explore which of these pedagogies can work effectively at
  massive scale or what we need to do to deal with scale (new services, metrics, and so on) in
  technology-enabled environment, while retaining their educational power and accessibility.
  Researchers have already recognized the importance of scale as a new first-class aspect of
  pedagogy, as evidenced by new scholarly conferences such as Learning At~Scale
  (\url{http://learningatscale.acm.org}).
\item[Rich because of scale] The new technological opportunities for multiple degrees of scale
  create new challenges, which should benefit from studies and lessons learnt regarding scale in
  telematics engineering, human networks, as well as networked learning. We therefore recommend that
  research efforts be directed to invent new pedagogical approaches that are intrinsically designed
  to take advantage of scale. We believe these two approaches of (a) scaling rich activities
  initially designed for low scale and (b) inventing new scale-powered activities, are complementary
  to the construction of new landscapes of learning through ambitious research programs.
\end{description}

\subsubsection*{Universities should move towards providing official credit for MOOCs}

In a nutshell, credits are necessary conditions for MOOCs to impact the European or worldwide
educational landscape.

Currently, most universities do not provide official credits towards an on-campus degree for MOOC
completion certificates.  This is especially important in Europe where credits are formally defined
in terms of learners workload and serve as an exchange currency, thereby allowing mobility among
curricula within an institution and among institutions. If MOOCs were recognized with official
credits, this current physical mobility would be enhanced with virtual mobility.  Physical plus
virtual mobility would both allow and motivate the construction of cross-institutional curricula,
something that already occurs in practice but is hampered by many institutional, technical and
logistical constraints.

\section{Quality assurance and measures of success}

\subsubsection*{Quality criteria for MOOCs should be developed}
Although the question of quality of a MOOC should, in principle, be no different than the question
of quality applied to any course produced by an educational provider, MOOCs have additional
properties that make the question of MOOC quality unique. First, MOOCs are usually free or very low
cost to the learner. Second, the massive scale has meant that the offering institutions have seen
MOOCs as a vehicle for publicity and reputational enhancement as much as an educational device.

\subsubsection*{Standards should be developed for measuring participation in MOOCs}

When governments, university administrators, instructors, learners, and the general public consider
a MOOC, measures of participation are critical in their perceptions and attitudes about the MOOC's
educational value. For example, when participation is measured in terms of number of videos watched
and assessments taken, there appears to be large attrition over time with virtually all MOOCs to
date, regardless of form or initial enrollment. Interpreted na\"{i}vely, these observations can be
taken as a negative aspect of MOOCs, but such measures fail to consider the goals of the learner and
other factors that might distinguish subpopulations with very different participation patterns.
Participation measures will inform assessments of (a) individual student learning outcomes, which is
perhaps primary. But through these assessments, participation measures will also inform assessment
of (b) cohorts of learners; (c) resources, the utility of which depends in part on use of these
resources, and the learner outcomes that result; (d) instructors, whose quality depends on the
outcomes of courses they teach; and (e) the quality of the MOOCs themselves and higher-level
administrative and educational structures.  As yet, we know of no consensus, detailed inventory of
variables for measuring participation in MOOCs, which could guide pedagogy, practice, and policy.
There is important research to be done on defining measures of participation and other educational
variables for MOOCs~\cite{deboer-ho-reconceptualizing}, studying the effect of these measures on the
perceptions of different stakeholders, and on how best to obtain and protect the learner data needed
for highly nuanced, conditionalized measures of participation.

\subsubsection*{Return on investment in MOOCs is heavily university-dependent}

MOOCs require considerable investment of capital and human resources.  Whether this investment will
yield effective return is heavily dependend on each institution's situation and priorities; a
corollary is that there is no single formula for computing return on investment or determining the
point of diminishing returns.

Universities operate in different contexts and so the opportunities open to any single university
may be different to those of others, due to differences in funding (fees vs.\ no fees), legal
frameworks (whether online learning is recognised as legitimate higher education), competiveness of
their student recruitment against other universities, and so on.  Each university must assess its
own possible ROI by balancing these constraints against the costs of developing and delivering
MOOCs.  A recent report~\cite{moocs-expectations-and-reality} offers a rich set of interview-based
perspectives on the costs and benefits of MOOCs as perceived and experienced by over 80 faculty,
administrators, researchers, and other actors from over 60 institutions in North America, Europe,
and China.

\section{Data access and privacy}

\subsubsection*{MOOCs should provide universal access}
By 2014 the world is expected to have more mobile-device accounts than it has inhabitants.  It is
therefore safe to assume that many learners will have them: user data from FutureLearn shows about
20\% of learners access the courses from mobile devices. 

Techniques such as Responsive Web Design can help provide an optimal viewing experience across a
variety of devices.  However, more important are the new affordances of these emerging devices, such
as smartwatches, smartbands, smartsensors, and so on.  Besides offering ``anytime, anywhere'' access
to learning materials such as podcasts and videos, mobile, wearable and ubiquitous technologies can
link physical and virtual environments and support continuous learner engagement through
notifications and instant messaging. In technology-enhanced learning, such devices can not only
provide information access, but also produce or collect content and communicate with each other.  We
therefore see an opportunity to provide continuity in time and space to promote seamless learning,
allowing online and technology-enhanced learning to ``break free'' of classroom constraints.


\subsubsection*{New data policies should be collectively negotiated}
MOOCs generate vast amounts of data: data about the content of the course (videos, quizzes,
exercises, slides, \ldots), data about the learners (clickstream data, answers to questions,
discussion forums, \ldots) and data about the professor and his or her pedagogical team. As any user
data collected by other online platforms, MOOCs raise serious concerns on possible breaches of the
user's privacy. 
      Consumer law, data privacy law and conventions vary by country and region.
      One possible position that is currently of interest is the
      principle that a learner’s data belongs to the learner.
Accordingly to this principle, learners should be able to easily access and visualize any data
recorded about them, such as interaction traces.  Learners should be able to share their data with
others as well as analyze it themselves, with access to the same analysis tools that MOOC platforms
or instructors use.  And a learner should be allowed to delete any subset of his data: he will lose
the advantages of having it analyzed, but that would be a personal choice.

But datasets do also constitute great opportunities. At the individual level, this data is necessary
for adapting learning activities to individual needs. At the collective level, the data supports
extracting knowledge about the effectiveness of the MOOC components, in order to improve the
concerned MOOC or to acquire general pedagogical knowledge. When sharing MOOC data one should be
attentive for the data to be accompanied by a semantic description of what was collected, that is,
what each piece of data meant in its original context. Such a description would also be necessary to
address the ethical issues already raised.

Finally, let us note that universities negotiate one-by-one with MOOC platforms, which puts them in
a rather weak position, despite the fact that they provide content that in has been developed over
many years with their own (often public) funding.  Universities and learners would be better served
if universities could move towards a collective negotiation, perhaps even at a national level for
countries that have specific privacy laws.  We therefore recommend that associations of universities
collect data privacy and data ownership concerns across their members institutions.

\section{Research questions}
Today a MOOC is primarily organized around lectures by a university instructor; it is taught as an
xMOOC in a traditional lecture-oriented format. One consequence is that MOOCs have a rather rigid
structure and suffer from high drop-out rates.  Future MOOCs should provide better learning
experiences; they should be more enjoyable, adaptive, evaluable, accessible, usable by handicapped
learners, aware of cultural diversity and ethical context, and so on.

These requirements create new research problems and new methods to address some of them including:
 
\begin{itemize}
\item \textbf{Better support for pedagogical methods}
\item \textbf{Better support for learner diversity}
\item \textbf{Better analytics of course data}

\end{itemize}

More research is also needed to understand the societal and economic impact of MOOCs. Interesting
research questions include ethics about data as well as learners or teachers behaviors, how to
transform on-line communities into geographically confined ones, and what MOOC business models will
balance the efforts of private industry and public services.


  % replace with whatever bib style shall be used
  \bibliographystyle{plain}
  \bibliography{manifesto}

\end{document}



%%% Local Variables: 
%%% mode: latex
%%% TeX-master: t
%%% End: 
