\section{MOOC Return on Investment is heavily university-dependent}
\label{sec:roi}

% editors: Jeff Haywood, Armando Fox
% participants: Yiwei Cao, Armando FOx, Jeff Haywood, Martin Wirsing,
%   Fran\c{c}ois Bry, Gilles Dowek

MOOCs require considerable investment of capital and human resources.
Whether this investment will yield effective return is heavily dependend
on each institution's situation and priorities; a corollary is that
there is no single formula for computing return on investment or
determining the point of diminishing returns.

Universities operate in different contexts and so the opportunities open
to any single university may be different to those of others, due to
differences in funding (fees vs.\ no fees), legal frameworks (whether online
learning is recognised as legitimate higher education), competiveness of their
student recruitment against other universities, and so on.  Each
university must assess its own possible ROI by balancing these constraints
against the costs of developing and delivering MOOCs.
 
We survey possible outcomes for a university of offering MOOCs---that
is, types of return they might hope to achieve---and then discuss
benefits specific to research vs.\ teaching universities as well as
possible negative returns on investment.

\subsection{Potential benefits and pitfalls of investing in MOOCs}
 
Universities may pursue MOOCs to meet any of several goals, including:

\begin{itemize}

\item Contributing to the wider mission of universities to reach more
potential learners for its education
\item  Enhancing the university's
reputation amongst key stakeholders, including politicians, peer
universities, potential students, and the media 
\item Improved teaching efficiency  by using MOOCs as part of courses
  for enrolled students 
\item Increased recruitment of students to credit-bearing and
  fee-bearing courses (where allowed)
\item Raising the university's
internationalisation profile  by engaging learners
worldwide
\item  Raising the university's outreach to those socioeconomically or
  geographically unable to
access residential higher education
\item Raising expectations for
on-campus teaching
%% , eg in terms of increased faculty productivity, or
%% modernising the format of teaching 
\item Training teachers and
students to work  in this new medium, as well as teaching future
professionals who need collaborative skills for lifelong learning 

\end{itemize}

However, MOOC investments are not without pitfalls.
At universities that rely heavily on tuition income, the considerable
expense of creating MOOCs may cut into other opportunities or
contribute to a negative perception that students' tuition money is
being ``wasted'' on giving away free content.
Enroled students, trustees, and funding agencies with such perceptions
may be indisposed to provide additional resources to these universities.
The attractiveness of investing in MOOCs may also draw funds away from less
innovative developments, and in particular from improving the quality of
existing but deficient-quality courses.
 
For all open-ended investments such as MOOCs, where the number of
investment opportunities is effectively unlimited, a very difficult
decision will be to decide at what point the return on investment moves
from positive to neutral or even to negative.  Can the same return be
obtained from 10 times or 100 times more MOOCs?
 
ROI can be improved if ways can be found to reduce the cost of
production, and the same applies for MOOCs.  Can these be produced more
cheaply (``MOOC factory'') or is the cost inevitably high due to the human
skill and time required to design and execute?


\subsection{Research universities versus teaching universities}

For research universities, MOOCs offer research materials and settings
for large scale research, with rapidly changing and evolving
settings. Research grants can flow from this opportunity.
As well, MOOCs can be a way of maximizing and accelerating the impact of
research, and may become a routine part of grant proposals, thus
incentivising universities to support the production as part of research
efforts MOOCs may become CV items for professors and junior faculty
giving them a direct return on their investment of time and effort.

For example, at Vanderbilt University in the US, the Institute for
Digital Learning provides matching support for faculty members to create a
MOOC if that MOOC is part of the proposed strategy for disseminating
research results, or is the major education component, in a successful research
proposal to the National
Science Foundation or other funding agencies, foundations, or industry sponsors.

%% Opportunities for researchers 
Beyond research on pedagogy and student engagement, numerous attractive
opportunities for research about the wider effects and context of MOOCs
are immediately evident.  For example, researchers might:

\begin{itemize}

\item investigate of the effectiveness of
MOOCs on outreach and access 

\item analyse the economics of MOOCs at all stages (creation, delivery,
re-use, cessation) and the value proposition of possible returns on
investment in a variety of contexts  \tbd{Armando: Cite Hollands report here}

\item investigate the impact of MOOCs on
``traditional'' university education and develop measures of those impacts
that can be used for comparative and longitudinal analyses 

\item gather
evidence from MOOC uses that can inform the effectiveness of policy
implementations, both during policy formulation and also policy
evaluation

\end{itemize}

What about the incentives for teaching-centric universities that
de-emphasize educational research?
At these institutions, informal explorations of teaching innovations
can still get the attention of others, and may lead to improved
teaching with higher retention and pass rates.  For example, in a widely
reported experiment at California's San Jose State University, students
taking a blended format electronics course using edX-provided materials
scored 5 to 6 percentage points better on exams, and 91\% of students
ultimately passed the course, compared with 59\% of students using the
conventional format only.  Of course, like research institutions,
teaching institutions should accept that not every experiment will be
equally successful: the same institution's attempt to provide remedial
maths courses through Udacity was less successful~\cite{cheal-sjsu-moocs}.

 
\subsection{How policymakers can help} 

MOOCs have potential benefits as
well as potential pitfalls for both research-intensive and
teaching-focused universities, and policymakers can help in several
ways:



\begin{itemize}

\item Be generally supportive of universities' efforts to develop MOOCs
  or reduce barriers 
to doing so where necessary, but with the understanding that bold
experiments are both necessary and risky.

\item As described in Section~\ref{sec:credit}, reduce barriers to
  universities offering credit for 
MOOCs \emph{should they wish to do so,} subject to normal quality assurance
processes.

\item Ensure that legal and compliance frameworks enable
  learners to use MOOCs for university entry and credit transfer, should
  universities wish to accept such credits.

\item Ensure that universities within your jurisdiction share knowledge,
expertise, and best practices, so that they may learn from each
others' mistakes as well as their successes.

\end{itemize}

We caution policymakers against mandating which options universities
must pursue with respect to MOOCs, at least without a careful advance
study of the complicated ecosystems in which universities operate,
the variety of factors listed above both supporting and opposing investment in
MOOCs, and the other dimensions of MOOC benefits and pitfalls discussed in this
report.  We similarly urge university leaders to take proactive roles in
such discussions rather than abdicating this responsibility.

