\section{MOOC ROI: Are MOOCs a good investment for universities?}

% editors: Jeff Haywood, Armando Fox
% participants: Yiwei Cao, Armando FOx, Jeff Haywood, Martin Wirsing,
%   Fran\c{c}ois Bry, Gilles Dowek

No such thing as general ROI; depends on university's
goals/context/constraints. Corollary: no single formula for diminishing
returns.  
 
Universities operate in different contexts and so the opportunities open
to any single university may be different to those of others, due to
differences in funding (fees vs.\ no fees), legal frameworks (whether online
learning is recognised as legitimate higher education), competiveness of their
student recruitment against other universities, and so on.  Each
university must assess its own possible ROI by balancing these constraints
against the costs of developing and delivering MOOCs.
 
The main possible outcomes for a university of offering MOOCs include:

\begin{itemize}

\item Contributing to the wider mission of universities to reach more
potential learners for its education
\item  Enhancing the university's
reputation amongst key stakeholders, including politicians, peer
universities, potential students, and the media 
\item Improved teaching efficiency  by using MOOCs as part of courses
  for enrolled students 
\item Increased recruitment of students to credit-bearing and
  fee-bearing courses (where allowed)
\item Raising the university's
internationalisation profile  by engaging learners
worldwide
\item  Raising the university's outreach to those socioeconomically or
  geographically unable to
access residential higher education
\item Raising expectations for
on-campus teaching
%% , eg in terms of increased faculty productivity, or
%% modernising the format of teaching 
\item Training teachers and
students to work  in this new medium, as well as teaching future
professionals who need collaborative skills for lifelong learning 

\end{itemize}

\subsection{Research universities vs. teaching universities}


For
research universities, MOOCs offer research
materials and settings for large scale research, with rapidly changing
and evolving settings. Research grants can flow from this opportunity.
As well, MOOCs can be a way of maximizing and accelerating the impact of
research, and may become a routine part of grant proposals, thus
incentivising universities to support the production as part of research
efforts MOOCs may become CV items for professors and junior faculty
giving them a direct return on their investment of time and effort.

%% Opportunities for researchers 
Investigation of the effectiveness of
MOOCs towards the wider goals of high education, namely outreach and
access Analyse the economics of MOOCs at all stages (creation, delivery,
re-use, cessation) and the value proposition of possible returns on
investment in a variety of contexts 

Investigate the impact of MOOCs on
``traditional'' university education and develop measures of those impacts
that can be used for comparative and longitudinal analyses Gather
evidence from MOOC uses that can inform the effectiveness of policy
implementations, both during policy formulation and also policy
evaluation


For teaching-centric universities that de-emphasize  educational research,
informal explorations of teaching innovations can still get the attention of
others and may also lead to improved
teaching with higher retention and pass rates.

\subsection{Negative returns}

MOOCs are not without pitfalls.
At universities that rely heavily on tuitiion income, the considerable
expense of creating MOOCs may cut into other opportunities or
contribute to a negative perception that students' tuition money is
being ``wasted'' on giving away free content.
Enrolled students, trustees, and funding agencies with such perceptions
may be indisposed to resource these universities further.
The attractiveness of investing in MOOCs may also draw funds away from less
innovative developments, and in particular from improving the quality of
existing but deficient-quality courses.
 
 
For all open-ended investments such as MOOCs, where the number of
investment opportunities is effectively unlimited, a very difficult
decision will be to decide at what point the return on investment moves
from positive to neutral or even to negative.  Can the same return be
obtained from 10x or 100x more MOOCs?
 
ROI can be improved if ways can be found to reduce the cost of
production, and the same applies for MOOCs.  Can these be produced more
cheaply (‘MOOC factory’) or is the cost inevitably high due to the human
skill and time required to design and execute?
 
\subsection{Conclusion: Recommendations for policy makers }

Summary: Understand ROI with respect to speific institution , and don't
be afraid to explore

Overall caveat: Decreasing cost of existing educational efforts
(decrease numerator or increase denominator)
-vs-
*entirely new* opportunities for learning whose ROI is hard to quantify
(like the story of the invention of xerography - it creates its own new
market) => Don't think entirely in terms of the *current* educational
costs \& benefits, as new markets \& opportunities will emerge.



MOOCs have potential benefits for both research-intensive and
teaching-focused universities, so policymakers should in general be
supportive of  universities' efforts to develop MOOCs
or reduce barriers to doing so where necessary.


 Ensure that universities are not
prevented from offering credit for MOOCs should they wish to do so,
subject to normal quality assurance processes 

Ensure sharing of
knowledge and expertise amongst universities in your jurisdiction Ensure
that legal and compliance frameworks enable learners to use MOOCs for
university entry and credit transfer
 

