\section{MOOC ROI: Are MOOCs a good investment for universities?}

% editors: Jeff Haywood, Armando Fox
% participants: Yiwei Cao, Armando FOx, Jeff Haywood, Martin Wirsing,
%   Fran\c{c}ois Bry, Gilles Dowek

Return on investment (ROI) of creating and delivering MOOCs for
universities
 
Universities operate in different contexts and so the opportunities open
to any single university may be different to those of others, due to
differences in funding (fees vs no fees), legal frameworks (eg online
learning not recognised as legitimate HE), competiveness of their
student recruitment against other universities, and so on.  Each
university must assess its own possible ROI using the information it has
about the costs of developing and delivering MOOCs internally versus
returns that it might make, aiming to maximise positive returns and
minimise negative returns.
 
 
The main possible outcomes for a university of offering MOOCs include:
Contributing to the wider mission of universities to reach more
potential learners for its education.  Enhancing the university’s
reputation amongst key stakeholders (including politicians, peer
universities, potential students, media etc) Higher efficiency of the
teaching process by using MOOCs as part of courses for enrolled students
Increased recruitment of students to credit-bearing courses (with
increased fee income where this is possible or allowed) Raising the
internationalisation profile of the university by engaging learners
worldwide Raising the outreach of the university to those not able to
access ‘normal’ higher education for reasons of disadvantage
(socio-economic, geographic, medical) Increasing the expectations for
on-campus teaching, eg in terms of increased faculty productivity, or
modernising the format of teaching A route to train teachers and
students to work collaboratively online, using eg peer evaluation,
socially mediated collaboration etc, leading to good professional skills
for future academics who will work in this new medium, and for other
professionals who need related skills for lifelong learning For
universities that do educational research, MOOCs offer research
materials and settings for large scale research, with rapidly changing
and evolving settings. Research grants can flow from this opportunity.
for universities that don't do educational research or do very little
research in general (eg teaching universities), their informal
explorations of teaching innovations can still get the attention of
others and result in good recognition and may also lead to improved
teaching with higher retention and pass rates For research-intensive
universities, MOOCs can be a way of maximizing/speeding impact of
research, and may become a routine part of grant proposals, thus
incentivising universities to support the production as part of research
efforts MOOCs may become CV items for professors and junior faculty
giving them a direct return on their investment of time and effort
 
Some possible negative returns also exist from offering MOOCs: They may
cut into other opportunities or contribute to a negative perception that
money is being "wasted" on giving away free content, expressed by
enrolled students, trustees, funding agencies in government The
attractiveness of investing in MOOCs may draw funds away from less
innovative developments, and in particular from rectifying existing poor
quality courses
 
 
For all open-ended investments such as MOOCs, where the number of
investment opportunities is effectively unlimited, a very difficult
decision will be to decide at what point the return on investment moves
from positive to neutral or even to negative.  Can the same return be
obtained from 10x or 100x more MOOCs?
 
ROI can be improved if ways can be found to reduce the cost of
production, and the same applies for MOOCs.  Can these be produced more
cheaply (‘MOOC factory’) or is the cost inevitably high due to the human
skill and time required to design and execute?
 
Recommendations for policy makers Support universities to develop MOOCs
or reduce barriers where these exist Ensure that universities are not
prevented from offering credit for MOOCs should they wish to do so,
subject to normal quality assurance processes Ensure sharing of
knowledge and expertise amongst universities in your jurisdiction Ensure
that legal and compliance frameworks enable learners to use MOOCs for
university entry and credit transfer
 
 
 
 
Opportunities for researchers Investigation of the effectiveness of
MOOCs towards the wider goals of high education, namely outreach and
access Analyse the economics of MOOCs at all stages (creation, delivery,
re-use, cessation) and the value proposition of possible returns on
investment in a variety of contexts Investigate the impact of MOOCs on
‘traditional’ university education and develop measures of those impacts
that can be used for comparative and longitudinal analyses Gather
evidence from MOOC uses that can inform the effectiveness of policy
implementations, both during policy formulation and also policy
evaluation
 
 
 
 


Benefits /investment of using MOOC Outreach of universities fulfilling
of missions for globalization social reputation social media

Productivity for research improvement via cheap teaching Internalization
by getting many international students

decrease costs


[Armando's notes start here]

Overall caveat: Decreasing cost of existing educational efforts
(decrease numerator or increase denominator)
-vs-
*entirely new* opportunities for learning whose ROI is hard to quantify
(like the story of the invention of xerography - it creates its own new
market) => Don't think entirely in terms of the *current* educational
costs \& benefits, as new markets \& opportunities will emerge.

No such thing as general ROI; depends on university's
goals/context/constraints. Corollary: no single formula for diminishing
returns.  Kinds of "return on investment": fulfilling of missions for
globalization/Internalization by getting many international students/
branding: (Establish reputation so students will want to come to your
university institution ) raises expectations for on-campus teaching
increased faculty productivity (potentially). eg Stanford runs some
courses "live" first quarter, and w/o lecturer remaining 3 quarters.
train teachers *and students* to work collaboratively - peer evaluation,
socially mediated collaboration, etc good professional skill, and teach
future academics (students) how to work in this new medium for MOOCs
that do not directly correspond to campus courses (eg - ALL Edinburgh
MOOCs), some kinds of ROI do not apply (eg improving efficiency of own
courses); whereas at (eg) Berkeley, efficiency-related ROI is entirely
about on-campus courses (if indeed that's seen as an important source of
ROI).  for universities that do education research, MOOCs give you a way
to do it with larger populations and on faster timescales and with the
potential to realize benefit faster to a larger audience.  for
universities that don't (eg teaching universities), their informal
explorations of teaching innovations can still get the attention of
others and result in good research.  In some cases, even in the absence
of rigorous experimental protocol, there can be a benefit (as occurred
with SJSU and edX) or a cost (as occurred with SJSU and Udacity) Impact
and outreach - for research-intensive universities, MOOCs can be a way
of maximizing/speeding impact of research.  eg, Edinburgh and Vanderbilt
both incentivize their proposal-writing faculty to add this as part of
proposal.  MOOCs may become CV items this way, which is another form of
ROI.  Will professors make job choices in part based on university's
willingness to commit resources to their MOOCs?  (ALready happening with
candidates at Edinburgh)


Pitfalls/downsides: at institutions that rely heavily on tuition
revenue, may cut into other opportunities or contribute to a negative
perception that students' tuition money is being "wasted" on giving away
free content how do you know when you've reached diminishing returns on
MOOC expenditures?  No matter how good MOOC tools/ecosystem become, the
expensive part of creating a great course is the instructional design
and effort put into preparing good teaching.  That cost will not go
down, though it can perhaps be amortized better.  Universities may be
unwilling to spend money to make a really good course even better if
they need the money to fix deficient courses.  eg, Dan Klein's AI course
was already among highest-rated in Berkeley EECS department, yet he
spent a lot of time and money improving it further for MOOC.

