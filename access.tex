\section{MOOCs should provide universal access}

%% Participants: Serge Garlatti, Carlos Delgado Kloos, Yannis Dimitriadis, Mike
%% Sharples, Marcus Specht

%% Section editor: Serge Garlatti

By 2014 the world is expected to have more mobile-device accounts than
it has inhabitants~\cite{siliconindia-mobile-device-growth}.
It is therefore safe to assume that many learners will have them: 
user data from FutureLearn shows about 20\% of access to its courses comes from mobile
devices. \tbd{need citation?}

Techniques such as Responsive Web Design can help provide an optimal
viewing experience across a variety of devices.
However, more important are the new affordances of these emerging
devices, such as  smartwatches, smartbands,
smartsensors, and so on.
Besides offering ``anytime, anywhere'' access to learning materials such
as podcasts and videos, mobile, wearable and ubiquitous technologies can
%% enable crowdsensing, crowdsourcing, crowdcommissioning, and linking
%% of physical and virtual environments. Furthermore, mobile devices can
link  physical and virtual environments and 
support continuous learner engagement through
notifications and instant messaging. In technology-enhanced learning,
such devices can not only provide information access, but
also produce or collect content and communicate with each other.
We therefore see an opportunity to provide continuity in time and
space to promote seamless learning, allowing online and
technology-enhanced learning to ``break free'' of classroom constraints.

We consider three categories of potential benefits to MOOCs from mobile
and ubiquitous 
computing: information access, production and
collection of data by learners, and communication.

\textbf{Information Access.}  Timely access to educational information is 
necessary no matter the learning type or
pedagogical approach. The widespread and growing use of mobile
technology to access learning materials means there is an imperative to
provide access to learning materials on these devices. But there is no
evidence that learning is enhanced by mobile
access, so the design challenge is to enable ubiquitous access without
impairing the learning experience.  

\textbf{Production and collection of
data by learners.}  
\tbd{ck: comment by fb: This (excellent) section should explicitely mention Human Computation and Citizen Science - for 
policymakrs not to miss the section's mesaage and for search engine to properly index the manifesto. }
There are several learning situations that are
facilitated by mobile and ubiquitous technologies. In contextual
learning and situated learning, different items of the context
information can be retrieved, processed, communicated and shared by
other learners. For example, the iSpot environment
(\url{ispotnature.org}) enables learners to make observations of
birds, animals, plants, and so on, sharing photos and initial
identifications online. Then others in the community, including
wildlife experts, provide additional information and a more accurate
identification. Such learner-created materials can be
especially important in MOOCs with large numbers of participants. 

Mobile devices enable the teacher to organise active
learning activities that rely on mobile devices, such as
cultural outings, field trips, and excursions to meet local residents.
For example, in a hypothetical MOOC on the
Renaissance in Florence, learners in the online MOOC environment might be
in contact with residents of the City of Florence, asking them questions
or commissioning ``localised investigations'' such as
photographing buildings or interviewing museum curators. In general, we
can use the expertise of the crowd for powerful learning from differing
perspectives and cultures. For example, in the FutureLearn course ``The
Secret Power 
of Brands'', participants from many countries
brought their differing cultural and national
perspectives to the discussions. Other possibilities include
crowd-sociology, crowd-psychology, and crowd-demographics. 
%% More
%% generally, learners could get multiple perspectives from different
%% cultures, topics (parents, teachers, students, etc.), disciplinaries,
%% etc.
Such situated learning can take place in inquiry-based learning, mastery
learning, case-based learning, problem-based learning, and project-based
learning, among other pedagogical approaches.

\textbf{Communication.} Mobile devices enable learners and other
stakeholders to have continuous social interactions and to be
continuously involved in their learning activity.
Participants can receive notifications during commuting time, exchange
short messages, stay abreast of the learning activities of their peers,
and so on.
Some classic pedagogies can be revisited with these affordances in
mind; for example, spaced repetition, which is especially good for
vocabulary language learning, could exploit mobile devices to deliver
reinforcers at timed intervals.

New opportunities are
possible in MOOCs from crowd learning, crowd sensing, crowdsourcing,
and crowd commissioning, all of which rely on data produced by ``the
masses.'' 
Crowd learning
describes the process of learning from the expertise and opinions of
others, shared through online social spaces, websites, and
activities~\cite{innovating-pedagogy-2013}. 
Crowd sensing is a new sensing
paradigm based on various mobile devices, wearable devices, connected
objects, and so on,  using these devices' sensors for crowdsourced
data creation and analysis that contribute to learning analytics with
authentic data and situations. Crowdsourcing  is a way to solicit  needed
services, ideas, or information
from a large group of people. Crowd Commissioning 
commissions learners to do tasks according to learning goals.  

MOOCs in different domains have already shown the promise of
%% importance and relevance of
authentic data and situated learning to give the learner a continuous
connection to a MOOC and to link virtual and physical spaces.
The access and creation of authentic data by crowds can enhance the
learning experience in MOOCs. 
These innovations, anchored firmly on a suitable pedagogical foundation,
can provide sustained motivation and personal relevance for learners.

\tbd{ck: comment by fb: In my humble opinion, MOOCs are also about a country's (and the European
  Union's) cultural reach in a globalized world anf, therefore, indirectly, an issue of considerable
  economical and political importance.

  I would suggest tthat he manifesto daring to mention this. }
