\section{MOOCs should provide universal access}

%% Participants: Serge Garlatti, Carlos Delgado Kloos, Yannis Dimitriadis, Mike
%% Sharples, Marcus Specht

%% Section editor: Serge Garlatti

Many modern web-based systems provide a `responsive' design that allows
material and services to be accessed on mobile and desktop devices, with
the aim of providing universal access. Where this responsive design has
been implemented for MOOCs, user data (FutureLearn) shows about 20\% of
learners access the courses from mobile devices.

Besides offering anytime, anywhere access to learning materials such as
podcasts and videos, mobile, wearable and ubiquitous technologies have
some additional affordances that enable successful MOOC design and
implementation. These include crowd-sensing, crowd-sourcing,
crowd-commissioning, and linking of physical and virtual
environments. Furthermore, mobile devices can support continuity of user
involvement and engagement through notification and instant
messaging. In technology-enhanced learning, mobile devices and wearable
technologies like connected objects (smartwatch, smartband, smartsensor,
etc.) provide universal access anytime anywhere to information, to
produce and/or collect content, and to communicate with each
others. Mobile devices can provide continuity in time and space to
promote seamless learning.

Therefore, we can analyse the benefits to MOOCs of mobile and ubiquitous
computing in three categories: information access, production and
collection of data by learners, and communication.

\textbf{Information Access.}  Timely access to educational information is a
necessary functionality for MOOCs, whatever the learning type and the
pedagogical approach. The widespread and growing use of mobile
technology to access learning materials means there is an imperative to
provide access to learning materials on these devices. But there is no
evidence that learning of delivered content is enhanced by mobile
access, so the design challenge is to enable ubiquitous access while not
greatly impairing the learning experience.  

\textbf{Production and collection of
data by learners.}  There are several learning situations that are
facilitated by mobile and ubiquitous technologies. In contextual
learning and situated learning, different items of the context
information can be retrieved, processed, communicated and shared by
other learners. For example, the iSpot environment
(\url{http://www.ispotnature.org}) enables people to make observations of
nature (birds, animals, plants etc), sharing photos and initial
identifications online. Then other people in the community, including
wildlife experts, provide additional information and a more accurate
identification. This feature of learner-created materials can be
especially important in the case of large numbers of participants, as in
the case of MOOCs. Mobile devices enable the teacher to organise active
learning activities that you cannot do without your mobile:
e.g. cultural sciences, field trips, excursions, meet the locals (like
dating services). For example, in a hypothetical MOOC course on the
Renaissance in Florence, people in the online MOOC environment might be
in contact with people in the City of Florence - asking questions of the
local residents or commissioning localised investigations, such as
photographing buildings or interviewing museum curators. In general, we
can use the expertise of the crowd for powerful learning from differing
perspectives and cultures. In the FutureLearn course ``The Secret Power
of Brands'', participants brought their differing cultural and national
perspectives to the discussions. Other possibilities include
crowd-sociology, crowd-psychology, and crowd-demographics. More
generally, learners could get multiple perspectives from different
cultures, topics (parents, teachers, students, etc.), disciplinaries,
etc.

 Learning in context and situated learning can take place, at least, in
the following pedagogical approaches: inquiry-based learning, mastery
learning, case-based learning, problem-based learning and project-based
learning.  

\textbf{Communication.} Mobile devices enable learners (and
more generally stakeholders) to have continuous social interactions and
to be continuously involved in their learning activity. They can receive
notifications, e.g. during commuting time, short messages (Instant
messaging is mostly done mobile), be aware of learning activities of
each others, etc. Some classic pedagogies can be revisited in a mobile
world, for example, spaced repetition
(\url{http://en.wikipedia.org/wiki/Spaced_repetition}) is especially good for
vocabulary language learning. You can take advantage of mobile devices
to deliver reinforcers at timed intervals.  

New opportunities are
possible in MOOCs from crowd learning, crowdsensing, Crowdsourcing,
Crowd-Commissioning. These opportunities are strongly related to the
production of data in mass and by the mass. Thus, crowd learning
describes the process of learning from the expertise and opinions of
others, shared through online social spaces, websites, and activities
(Innovating Pedagogy 2013,
\url{http://www.open.ac.uk/blogs/innovating/}). Crowd sensing is a new sensing
paradigm based on various mobile devices, wearable devices, connected
objects, etc. it is using sensors of these devices for crowd-sourced
data creation and analysis that contribute to learning analytics with
authentic data and situations. Crowdsourcing is is a way to get needed
services, ideas, or information, content by soliciting contributions
from a large group of people. Crowd Commissioning can be the practice to
commission learners to do tasks according to learning goals.  

Nowadays,
providing universal access is required in MOOCs as most of learners are
using mobile devices. MOOCs in different domains have shown the
importance and relevance of authentic data and situated learning for
sustainable motivation and personal relevance (citations of Examples,
iSpot, medical MOOC, Cultural excursions, marketing course on brands,
\ldots). The access and creation of authentic data in mass (for instance mass
production of evidence) is a key issue to enhance the learning
experience and the acquired knowledge and skills, in MOOCs. Moreover,
universall access enable the learner to feel continually connected to a
MOOC and to combine virtual and physical spaces.
