\section{Rich despite scale, or rich because of scale?  Scaling up pedagogy}

%% Section editors: Marcus Specht, Pierre Dillenbourg - Revised 22 mars

MOOCs offer new opportunities for scaling up education to very large
numbers of students:
they can reach up to 200.000 students, compared with the approximately
500 in a typical lecture or seminar room.
This scale is made possible
in part by the ``open'' features of MOOCs and their worldwide
accessibility.
The drawback is that scaling up could come at the
expense of pedagogical richness. Typically, the range of pedagogies for
xMOOCs is limited primarily to delivery of content and computer-based
assessment. This may include effective learning tools such as
simulations or modelling tools, as well as peer-review and group
work. Nonetheless, today we see scale primarily as a limitation: some
rich learning 
activities are difficult to scale up. In this section we consider
how to use these rich methods despite scale, and we explore a
complementary perspective---that some activities work better when conducted at
large scale. In these cases scale is an opportunity.  We therefore
divide our discussion into MOOC elements that are \emph{rich despite
  scale} and those that are \emph{rich because of scale}.

\subsection{Rich despite scale}

An example of pedagogical approach that is difficult to scale is small
group activities oriented to joint production of artifacts like essays
or design documents in project-based learning, as well their interplay
with large group activities. Which categories of pedagogical methods do
not scale up very well? While there are many categorizations of
pedagogies for online learning, a framework developed for the Capital
project can be considered as a valid reference. The modes of learning
are described by multiple keywords, such as reflective, collaborative,
simulation, construction, inquiry-driven, problem-based, project-based,
case-based, cross-context, game-based, anchored, performative,
conversational, networked, and embodied. Each of these methods is
mediated in practice through location, process, topic, technology,
representation and actors. So, for example, inquiry-driven learning
might be enacted at home, by self-directed processes, using mobile,
ubiquitous, and pervasive technology, with a representation of the
inquiry process, through communicating peers.

\tbd{ Doug Fisher: what is difference between ubiquitous and pervasive
 here?  (Serge Garlatti suggests using 'mobile' instead?)}


In the space of pedagogical methods, those that are difficult to conduct
at scale tend to be those that scaffold high order thinking skills or
competencies such as creativity, critical thinking, collaboration
skills, and scientific rigour.  The importance of these skills explains
why we care about pedagogical diversity at scale and leads us to the
following \textbf{recommendation expressed as a research question:} How can we
create a broad range of effective pedagogies at massive scale, and
efficiently contribute to achieve 21st century competencies? The
challenge is to explore which of these pedagogies can work effectively
at massive scale or what we need to do to deal with scale (new services,
metrics, etc.) in technology-enabled environment, while retaining their
educational power and accessibility.

Research should focus on the orchestration functions that heavily depend
on scale. These functions include creating breakout groups, delivering
extra content and personal coaching as in SPOCs, monitoring progress,
surveying student status, guiding and scaffolding the different learning
activities, peer reviewing or peer assessment. Thus, one would need to
provide orchestration and analytic methods for educators to manage
important problems of scale for purposes of preparation, monitoring,
grouping, surveying, directing, and coaching. The ultimate goal would be
to reduce and distribute efficiently the associated orchestration load
on educators, learners, and software agents. Then, orchestration
bottlenecks should be identified and effective load balancing schemes
should be provided. An illustrating analogy consists in the functions of
the human or virtual game masters (i.e., community managers) in modern
massive multiplayer games.

Similarly, scaling down can be also very relevant, as when the number of
participants decreases drastically within the lifetime of an on-line
course, or when a massive course has to adopted and adapted for another
context of lower scale. Moreover, massive scale can sometimes be best
achieved by aggregating a massive number of small learning cohorts,
again highlighting the importance of small group dynamics and the
importance of scale-down,
as discussed in Section~2.2 of \cite{mroe-2013-report}.
Therefore, there is a need for flexible and
elastic mechanisms that enable the most effective and efficient
pedagogies when scaling up and down.  

\subsection{Rich because of scale}

To understand the effects of massive scale, we can call on the
Metcalfe’s Law which states that the value of a networked system
increases in proportion to the square of the number of connected users
of the system. For example, a telephone system becomes increasingly
effective with the number of subscribers able to communicate. But as
Stephen Downes indicates
(\url{http://halfanhour.blogspot.de/2007/11/personal-network-effect.html}),
learners are not simply nodes in a communications network, they engage
in meaningful discussion and collaboration. Such interactions should be
useful, new, salient, timely, understandable, trusted, and coherent. The
new technological opportunities for multiple degrees of scale create new
challenges, which should benefit from studies and lessons learnt
regarding scale in telematics engineering, human networks, as well as
networked learning.

The mere existence of a scaled up number of learners may have a direct
effect on the scale of interactions, and especially on the quantity and
quality of messages and other emerging learning objects which are
produced and propagated through the networked participants (learners,
teachers and other stakeholders). Also, the larger the network, the more
opportunities there are for direct learner-to-learner communication. For
instance, students involved in reflective learning may benefit from
being aware of the annotations of large numbers of other co-learners,
observe the heat maps of these annotations, and consequently focus on
the critical topics. However, such a major opportunity requires the
provision of adequate filters and advanced semantic and lexical
processing methods, that will make peer annotations meaningful. Other
significant modes such as collaborative learning benefit from the
crowdsourcing effects, especially in terms of timely peer reviews, or
data analytics using big data can feed effective algorithms for
appropriate groupings.

We therefore \textbf{recommend} that research efforts be directed to invent new
pedagogical approaches that are intrinsically designed to take advantage
of scale. We believe these two approaches of (a) scaling rich activities
initially designed for low scale and (b) inventing new scale-powered
activities, are complementary to the construction of new landscapes of
learning through ambitious research programs.
