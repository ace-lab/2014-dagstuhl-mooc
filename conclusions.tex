\section{Conclusions}

\tbd{Summarize recommendations/theses here??}\\
\tbd{CK: I don't think so, they are well presented in the ex
summary}

%% inspirational closing words


The invention of the printing press led to a technical-intellectual ecosystem consisting of print
shops, skilled laborers, authors, editors, readers, publishers, and investors, motivated by both
profit and intellectual pride~\cite{febvre}.  The universities that successfully embraced this
disruptive technology---Frankfurt, Paris, Strasbourg, Leiden, Venice---became the most influential
of their time, as the ideas and writings of their faculty gained wide visibility.

Similarly, we believe modern universities must embrace the disruptive technology of MOOCs as
vigorously as European Renaissance universities embraced printing to enhance and cement their
intellectual leadership.  Like the printing press, MOOCs provide not only new financial
opportunities,
%% not only to make a profit, but also but also new ways to spread ideas
but new ways to enhance reputation.  They can help attract the best students and faculty, provide
them with modern learning environments, and in so doing, contribute to the success of both learners
and institutions.  These characteristics will make MOOCs an essential component of success and
visibility in today's higher education.  Universities wishing to retain their intellectual
prominence ignore this technology at their peril.

Finally, we close with an observation.  2013 marked the 75th anniversary of xerography, the
dry-photoimaging process that revolutionized business. In 1938, few believed that xerography would
ever displace carbon paper, a time-tested and extremely inexpensive way of making multiple
originals.  As in many similar examples, they did not see that the fundamentally new ability to
quickly copy \emph{existing} documents would enable entirely new markets and practices that were
literally inconceivable with carbon paper.

We similarly caution university leaders and policy makers not to limit their appraisal of MOOCs to
comparison with existing techniques only.  New educational modalities, markets, and opportunities
will arise that were previously inconceivable.  Each university should be empowered to explore and
seize the opportunities that seem most consistent with its institutional situation and goals, but
with the realistic understanding that not every experiment will end in success, and that failures
should be regarded as opportunities for learning rather than reasons to turn back.


%%% Local Variables: 
%%% mode: latex
%%% TeX-master: "manifesto"
%%% End: 
