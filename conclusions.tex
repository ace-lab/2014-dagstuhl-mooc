\section{Conclusions}

\tbd{Summarize recommendations/theses here??}

%% inspirational closing words


The invention of the printing press led to a
technical-intellectual ecosystem consisting of print shops, skilled
laborers, authors, editors, readers, publishers, and investors,
motivated by both profit and intellectual pride~\cite{febvre}.
The universities that successfully embraced this disruptive
technology---Frankfurt, Paris, 
Strasbourg, Leiden, Venice---became the most influential of their
time, as the ideas and writings of their faculty gained wide visibility.

We conclude that modern universities must embrace the disruptive
technology of MOOCs
as vigorously as European Renaissance universities
embraced printing to enhance and cement the
visibility of their intellectual leadership.
Like the printing press, 
MOOCs provide  not only new financial opportunities,
%% not only to make a profit, but also
but also new ways to spread ideas and gain reputation. 
They will help attract the best students and
faculty, provide them with modern learning environments, and in so doing,
contribute to the success of learners and institutions.
These characteristics will make MOOCs an
essential component of success and visibility in today's higher
education. 
Universities wishing to retain their intellectual
prominence ignore this technology at their peril.




