\begin{abstract}

The rapid emergence and adoption of MOOCs (Massive Open
Online Course) has raised new questions and rekindled existing debates
in higher education.
Academic leaders are concerned about cost management, educational
quality, access to content, privacy protection for learner data, and the
proper relationship between MOOCs and residential instruction, among
other matters.
At the same time, these same leaders see opportunities for the scale of MOOCs
to support learning rather than thwart it: better
learner engagement through personalization, increased understanding of
student behavior through large-scale data analytics, wider access
for continuing education students and other nonresidential learners, and
the possibility to enhance revenue or lower educational costs.
Two years after ``the year of the MOOC'', this report summarizes the
state of the art and the future directions of greatest interest as seen
by an international group of academic leaders,
and makes eight recommendations to policy makers,
academics, administrators, and students regarding the potential future
of MOOCs in higher education.  These recommendations span a variety of
topics including financial considerations, pedagogical quality, and
the social fabric.

\end{abstract}


