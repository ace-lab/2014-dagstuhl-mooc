\section{Universities should refactor residential education using MOOCs}

% editor: Armando Fox
% participants: Fran\c{c}ois Bry, Carlos Delgado Kloos, Jens Dittrich,
% Armando Fox, Martin Wirsing, Armin Weinberger


The MOOC-accelerated paradigm shift in education, comparable to shifts
in the news media and the music industry,
is forcing an unbundling and rebundling of educational components. In the
case of higher education, 
we must rethink long-established modes of interaction, teaching roles,
and structures in a way that recognizes
that the sources of knowledge and therefore the roles of instructors and
campuses have changed.

We see the most important shifts as follows:

\begin{itemize}

\item With the wide availability of high-quality, (often) free, and
easy-to-reuse content online, most instructors' primary responsibility
shifts \emph{from creating content to creating context}.

\item Rearrangements of residential course elements (lectures,
  recitations, labs, and so on) and new teaching roles beyond simply ``instructor''
  and ``teaching assistant'' will better fit this new instructional
  modality.  One challenge of this reorganization is changes to campus
  infrastructure, but one facilitator is that inexpensive video production
  will empower domain experts to create 
  better materials to prepare and train the new teaching roles.

\item The absence of the ``sage on the stage'' will open new ways to
foster teacher and learner commitment.  Campuses should
focus less on conveying content-oriented skills and more on
social/professional skills, such as collaborative work and
perspective-broadening activities, to complement  independent study
and discovery.

\end{itemize} 

We conclude that modern universities must embrace this disruptigve
technology ecosystem as vigorously as European Renaissance universities
embraced mechanized printing to enhance and cement the
visibility of their intellectual leadership.


\subsection{Instructor responsibilities: Context vs Content}

The abundance of learning material online---instructional videos,
slides, discussion forums, open educational resources,
Wikipedia---challenges the traditional role of teachers and lecturers as
content creators, a role rooted in times when such materials were not
easily available.
Today, lecturer-authored content competes with hundreds of alternatives,
some better and some worse than his own, and the lecturer
loses the content monopoly he has had since classical times.
MOOC-driven learning materials may supplant lectures for the first two
learning steps in Bloom's taxonomy:  remembering and understanding the content
presented by the lecturer.

In this new scenario, the instructor's role focuses on contextualizing
existing content.
Numerous metaphors have been used to describe this role: creating a
narrative, connecting the dots, acting as an expert tour guide, or as in
Vannevar Bush's prescient Memex vision, blazing a trail.
The instructor presents his or her personal view on the content based on
domain expertise, puts it into perspective with examples, applications,
analogies or anything else he/she deems useful. The lecturer questions
the content's assumptions and perspectives, helping students evaluate
alternative explanations of content and thereby guiding them in applying
the material (the third step in Bloom's taxonomy).
This role is especially important across disciplines: a course on
``building user-friendly and secure databases'' requires interweaving
expertise in databases, human-computer interaction, and security.
The abundance of online material in each topic suggests that with a
skilled guide, such a course could be offered without requiring three different
professors to teach it.
 
One key ingredient of this new role is the ability to 
skillfully select and remix existing content into a coherent story,
as a film editor reduces hundreds of hours of film footage into a
coherent 90-minute movie presenting a specific point of view.
We note, however, that this activity brings the lecturer's role into direct
conflict with restrictions on remixing imposed by copyright issues, as
discussed in Right to Remix (http://right2remix.org) and other forums.

Interestingly, this instructional format---instructor guidance around
inquiry-based use of existing materials---is the current model for
graduate research.  While undergraduates would be exploring an existing
body of knowledge rather than discovering new knowledge, it is possible
that the undergraduate educational process could become more like
research and less like the unidirectional presentation of information
that dominates undergraduate education today.


\subsection{Inexpensive video creation and the reorganization of courses
  and teaching roles}

Most residential courses today are offered in a ``one size fits all''
model: a single lecture, a single set of assignments or labs, and so on.
Yet as demand for higher education drives enrolments, we observe that
not all aspects of offering a course scale equally well in terms of
instructor resources.
Larger enrolments expose more obvious variance across student cohorts.
Combining flexible teaching staff with MOOC-like ``self-service''
resources can help tailor instruction to smaller groups within the large
cohort.   For example, peer and social learning techniques enable
strong students to help their colleagues, while struggling students 
benefit from mastery learning and other MOOC benefits resulting from
inexpensive computing power.
This refactoring of course elements into more-scalable and less-scalable
components
will leave instructors more time to conduct
interaction-intensive learning activities such as small-group 
discussions and design projects.

At the other extreme, MOOCs could provide an opportunity to save
highly-specialized ``boutique'' courses and curricula.
In Europe, after the Bologna Declaration, highly specialized courses and
curricula such as Indian Studies or Albanology (the study of Albania)
had to be integrated into more general Bachelors' or Masters' curricula
such as Asian Studies or Mediterranean Studies, threatening the
existence of important building blocks for certain cultures.
Several professors around the world could collaboratively produce a
series of specialized MOOCs that would form the basis of ``boutique''
degree programs that would enable continued deep scholarship in such
areas despite limited resources at any single university.

Both scenarios call for new teaching roles beyond just the traditional
roles of instructor and teaching assistant as they exist today.
How will these new roles receive training in order to be effective?
It is already well known that some skills are better demonstrated by
video than explained in text; two examples are narrated screencasts
showing how to use digital tools and videos showing how to play an
instrument or perform physical tasks.
Inexpensive video creation and distribution allows the domain expert to
author such materials directly, eliminating the ``translation gap'' that
might arise when working with a video producer who is not a domain
expert and thereby accelerating the creation and deployment of these
assets.
The potential benefit of this change is the ability to easily create
training materials \emph{for other instructors} that are richer and more
interactive than a textbook, such as a
SPOC (Small Private Online Course) targeted not at students but at other
instructors becoming involved with a course.
These materials could familiarize staff with particular course topics,
give guidance on resolving common student problems, and so on.
By exploiting the ability to create such materials, it becomes possible
to train new strata of teaching staff that further leverage the
effectiveness of the lead instructor, potentially allowing us to educate
more students with a sublinear increase in instructor resources.

\subsection{Teacher and learner commitment in new learning modes}

Recent meta-analyses (Hattie, 2009) have examind what makes teaching
effective and how the best teachers activate learners with feedback,
challenging goals, direct instruction, and frequent testing, acting as
behavioral organizers rather than facilitating learners to digest
learning material and fulfill tasks.
Just as the 2-sigma finding of tutorial instruction has inspired
intelligent cognitive tutors, these meta-analyses may inspire the
improvement of MOOCs or
whatever follows them.

How can we motivate MOOC learners and teachers when the teachers may be
both geographically distant and socially disconnected from the learners?
One positive observation is that the delivery vehicle for
MOOCs---large-scale software-as-a-service---allows 
research-supported best practices to be quickly integrated into MOOC software
and deployed to teachers and learners.
Examples could include MOOC support for 
reciprocal teaching, direct instruction, mastery learning, peer
assessment and instruction,
small-group/community interactions such as dynamic regrouping of
students to match learning styles and paces, and so on.

Campuses should therefore capitalize on their social and professional
networking benefits, teaching skills that are less content-oriented and
more crosscutting such as teamwork and collaboration.
Campuses should also focus on their ability to competitively place
students in perspective-broadening
activities, such as premium internships or international collaborations.

Finally, campus infrastructure today, both physical and electronic, is
highly optimized around traditional delivery models.
Campuses may have to rethink facilities in light of the increasing role
of personal devices such as 
smartphones and tablets, the greater importance for social learning and
interaction, and the increasing use of collaborative spaces in
preferences to computer labs and individual study spaces.

