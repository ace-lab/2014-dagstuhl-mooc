\section{Universities should refactor residential education using MOOCs}

% editor: Armando Fox
% participants: Fran\c{c}ois Bry, Carlos Delgado Kloos, Jens Dittrich,
% Armando Fox, Martin Wirsing, Armin Weinberger


We are experiencing a paradigm shift in education comparable to those
that have occurred in the news media and the music industry.   Such
shifts force an unbundling and rebundling of components. In the
case of higher education, and in particular undergraduate education,
we must rethink long-established modes of interaction, teaching roles,
and structures in a way that recognizes
that the sources of knowledge and therefore the roles of instructors and
campuses have changed.
MOOCs are playing a role similar to that of textbooks after the
emergence of print, but the content must be adapted to the local circumstances,
the place in the curriculum, prior knowledge, and so on.

We see the most important shifts as follows:

\begin{itemize}

\item With the wide availability of high-quality, (often) free, and
easy-to-reuse content online, most instructors' primary responsibility
shifts \emph{from creating content to creating context}.

\item Rearrangements of residential course elements (lectures,
  recitations, labs, and so on) and new teaching roles beyond simply ``instructor''
  and ``teaching assistant'' will better fit this new instructional
  modality.  One challenge of this reorganization is changes to campus
  infrastructure, but one facilitator is that inexpensive video production
  will empower domain experts to create 
  better materials to prepare and train the new teaching roles.

\item The absence of the ``sage on the stage'' will open new ways to
foster teacher and learner commitment.  Campuses should
focus less on conveying content-oriented skills and more on
social/professional skills, such as collaborative work and
perspective-broadening activities, to complement  independent study
and discovery.

\end{itemize} 

We conclude that modern universities must embrace this disruptigve
technology ecosystem as vigorously as European Renaissance universities
embraced mechanized printing to enhance and cement the
visibility of their intellectual leadership.


\subsection{Instructor responsibilities: Context vs Content}

The abundance of learning material online---instructional videos,
slides, discussion forums, open educational resources,
Wikipedia---challenges the traditional role of teachers and lecturers as
content creators, a role rooted in times when such materials were not
easily available.
Today, lecturer-authored content competes with hundreds of alternatives,
some better and some worse than his own, and the lecturer
loses the content monopoly he has had since classical times.

Students in a traditional lecture first need to complete the first two
learning steps in Bloom's taxonomy: remember and understand the content
presented by the lecturer.
Today, MOOC-driven learning materials used in blended teaching formats
like Flipped or Inverted Classroom provide greater opportunity for
self-study.
In this scenario, the lecturer's role focuses on contextualizing
existing content.
Numerous metaphors have been used to describe this role: weaving a
narrative through existing content, connecting the dots, acting as an
expert tour guide, or as in Vannevar Bush's prescient Memex vision,
blazing a trail.
The lecturer presents his or her personal view on the content based on
domain expertise, puts it into perspective with examples, applications,
analogies or anything else he/she deems useful. The lecturer questions
the content's assumptions and perspectives, helping students evaluate
alternative explanations of content and thereby guiding them in applying
the material (the third step in Bloom’s taxonomy).

The lecturer therefore mediates between  the students and
a ``tsunami'' of unconnected content, selecting 
and remixing it into  a coherent story analogously to how
a film editor reduces hundreds of hours of film footage into a coherent
90-minute movie.
This potentially brings the lecturer's role into direct conflict
with restrictions on remixing imposed by copyright issues, as
discussed in Right to Remix (http://right2remix.org) and other

The role of lecturers as guides applies across disciplines as well as within a specialty.  For example, ``building a user-friendly and secure
database''  crosses at least three different Computer Science specialties
(databases, human-computer interaction, and security). 
This kind of interdisciplinary
story-weaving is facilitated by the abundance of online material which
could be recomposed and selected for this kind of course---perhaps even
without requiring three different professors to teach it.

Interestingly, this instructional format---instructor guidance around
inquiry-based use of existing materials---is the current model for
graduate research.  While undergraduates would be exploring an existing
body of knowledge rather than discovering new knowledge, it is possible
that the undergraduate educational process could become more like
research and less like the unidirectional presentation of information
that dominates undergraduate education today.


\subsection{Course reorganization, new teaching roles,  and the impact of
  inexpensive content creation} 

Most residential courses today are offered in a ``one size fits all''
model: a single lecture, a single set of assignments or labs that
everyone completes, multiple recitation sections that generally cover
the same material as one another, and so on.
Yet as demand for education drives higher enrollments, we observe that
not all aspects of offering a course scale equally well in terms of
instructor resources.
Larger enrollments expose more obvious variance across student cohorts;
combining flexible teaching staff with MOOC-like ``self service''
resources can help tailor instruction to smaller groups within the large
cohort, with peer-learning techniques introduced so that 
strong students can help their colleagues and struggling students can
benefit from mastery learning and other MOOC benefits resulting from
inexpensive computing power.
This refactoring can leave instructors more time to conduct
interaction-intensive learning activities such as small-group 
discussions and design projects.
In general, MOOCs present an opportunity to ``refactor'' large residential
courses into more-scalable and less-scalable components.

At the other extreme, MOOCs could provide an opportunity to save small
``boutique'' courses and curricula.
In Europe, after the Bologna Declaration, highly specialized courses and
curricula such as Indian Studies or Albanology had to be integrated into
more general Bachelors' or Masters' curricula such as Asian Studies or
Mediterranean Studies, threatening the existence of important building
blocks for certain cultures.
Several professors around the world could collaboratively produce a
series of specialized MOOCs that would form the basis of ``boutique''
degree programs that would enable continued deep scholarship in such
areas despite limited resources at any single university.

Both scenarios call for new teaching roles beyond just the traditional
roles of instructor and teaching assistant as they exist today.
How will these new roles receive training in order to be effective?
One opportunity is that video production and distribution are now
sufficiently inexpensive that individual instructors can create these
assets on their own, rather than requiring hand-holding from an
expensive production team.
It is already well known that some skills are better demonstrated by
video than explained in text; two examples are narrated screencasts
showing how to use digital tools and videos showing how to play an
instrument or perform physical tasks.
Inexpensive video creation and distribution allows the domain expert to
also be the author of such materials, eliminating the ``translation
gap'' that might arise when working with a video producer who is not a
domain expert and accelerating the creation and deployment of these
assets.
The potential benefit is the ability to easily create training materials
\emph{for other instructors} that are richer and more interactive than a
textbook. For example, an instructor might create a SPOC (Small Private
Online Course) targeted not at students but at other instructors or at
teaching assistants.
These materials would help train teaching staff to assist students in
specific ways: familiarizing staff with a particular homework
assignment and common student problems, giving staff suggestions for
reviewing material before an exam, giving them advice on how to handle
problematic situations in the classroom, and so on.
By exploiting the ability to create such materials, it becomes
possible to train new strata of teaching staff that further leverage
the effectiveness of the lead instructor, potentially allowing us to
educate more students with a sublinear increase in instructor
resources.


\subsection{Teacher and learner commitment in social learning}

Recent meta-analyses (Hattie, 2009) have examind what makes teaching
effective and how the best teachers activate learners with feedback,
challenging goals, direct instruction, and frequent testing, acting as
behavioral organizers rather than facilitating learners to digest
learning material and fulfill tasks.
Just as the 2-sigma finding of tutorial instruction has inspired
intelligent cognitive tutors, these meta-analyses may inspire the
improvement of MOOCs or
whatever follows them.

An inherent challenge of MOOCs is
that learners cannot easily be personally involved, committed, and
activated to a productive learning trajectory. This may not only be a
problem of lacking meta-cognitive skills for self-regulation, but
of different cultural understandings of what education and  courses are, as
well as the model of learning and instruction that is inherent to many
MOOCs, i.e. transmission of information bits with a teacher who may be
both geographically distant and socially disconnected from her students.

To address this challenge, educational technology
approaches may be: provide for introductions that include compressed
assessment, self-positioning/introductions and trainings in
meta-cognitive strategies for online learning of participants develop
compiled forms of automated and peer assessment that can be used
continuously and fed back to students instantly, but also aim for
summative evaluation provide computer-supported, ready-to-use scripts
for teachers and learners that build on a robust body of evidence, e.g.,
reciprocal teaching, direct instruction, mastery learning provide roles
to students and teachers adaptively orchestrate learning activities in
different learning arrangements from individual to small group to
community interactions including dynamic regrouping of students arrive
at new rhythms for learning (breaks, inverted classrooms, independent
study time, micro learning, time for argumentation) cater for
connections from MOOC to local courses to online presences of students

 Should campuses capitalize on their social and professional
  networking benefits, teaching skills that are less content-oriented
  and more crosscutting such as teamwork and collaboration and focusing
  on placing students in internships or international experiences that
  will broaden their perspectives?

 Campus infrastructure today is highly optimized around traditional
  delivery models.  What is
the role of personal devices such as smartphones and tablets for social
learning interaction? Should campuses 
  have fewer computer labs and more open space for collaboration?  


\subsection{Summary: MOOCs and the intellectual prominence of universities}

The invention of the printing press led to a  technical-intellectual
ecosystem consisting of print shops, skilled laborers,
authors, editors, readers, publishers, and investors, motivated by both
profit and intellectual pride~\cite{febvre}.
The universities surrounded by such ecosystems---Frankfurt, Paris,
Strasbourg, Leiden, Venice---became the most influential of their
time, as the ideas and writings of their faculty gained wide visibility.

Like the printing press, 
MOOCs provide an opportunity not only to make a profit, but also
to spread ideas and gain reputation. They will therefore be an
essential vector of success and visibility in today's higher
education. They will help attract the best students and
faculty, provide them with modern learning environments, and in so doing,
contribute to the success of learners and institutions.
Universities wishing to retain their intellectual
prominence ignore this technology at their peril.



