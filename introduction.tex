\section{Introduction and Executive Summary}

% section editor: Martin Wirsing – March 23

Online education is not new; Massively Open Online Courses (MOOCs)
are. Their uniquely powerful combination of classical digital teaching
tools (videos, audios, graphics or slides), individualized tools for
acquiring and validating knowledge, and appropriate use of dedicated
social networks makes them a new and formidable means of accessing
knowledge and education. If backed up with scientific and pedagogical
excellence, MOOCs allow one to reach and teach simultaneously tens of
thousands and even hundreds of thousands of learners in a new
pedagogical dynamic.

Of the numerous MOOCs initiatives that have recently emerged, especially
in the US and Europe, a few seem to be surfacing with an extremely
important impact. This creates a very new situation and indeed can be
considered as the informatics community's first main impact on knowledge
dissemination and teaching. MOOCs will very likely induce a radical
change in teaching mechanisms and their links to the economic and
production systems. The consequences with respect to the transmission of
culture and educational content, and on society as a whole, will be
deep.

The Perspectives Workshop on ``Massively Open Online Courses, Current
State and Perspectives'' took place at Schloss Dagstuhl on March 10--13,
2014. More than 20 leading researchers and practitioners from
informatics and pedagogical sciences presented and discussed current
experiences and future directions, challenges, and visions for the
influence of MOOCs on university teaching and learning.  The first day
of the workshop consisted of a series of presentations in which each
participant presented those topics and developments he or she considered
most relevant for the future development of MOOCs. On the second and
third day the participants divided into several working groups according
to the main thematic areas that had been identified on the first day.

%% [Old: The current discussions about MOOCs encompass many questions on
%% the pedagogical engineering of MOOCs, economical models, ethical issues,
%% the technical development of platforms, and sharing. The participants
%% decided to focus on eight theses which can be divided into three
%% categories: the integration of MOOCs into university education, quality
%% assurance and measures of success, and policies for access and privacy.]

This manifesto summarizes the key findings of the workshop and provides
a collection of research topics for MOOCs. The eight theses presented in
this report can be divided into three categories: the integration of
MOOCs into university education, quality assurance and measures of
success, and policies for access and privacy.

\vspace{1ex}
\textbf{Integration of MOOCs into university education}

\begin{enumerate}

\item Universities should refactor residential education using
MOOCs

\item New pedagogical approaches should be developed to take
advantage of scale

\item Universities should move towards providing official
credit for MOOCs

\end{enumerate}

\textbf{Quality assurance and measures of success}

\begin{enumerate}
\setcounter{enumi}{3}

\item Quality criteria for MOOCs should be developed 

\item Standards should be developed for measuring
  participation in MOOCs 

\item Return on investment in MOOCs is heavily university-dependent

\end{enumerate}

\textbf{Data access and privacy}

\begin{enumerate}
\setcounter{enumi}{6}

\item MOOCs should provide universal access 

\item New data
policies should be collectively negotiated

\end{enumerate}
